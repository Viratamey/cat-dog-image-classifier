
% Default to the notebook output style

    


% Inherit from the specified cell style.




    
\documentclass[11pt]{article}

    
    
    \usepackage[T1]{fontenc}
    % Nicer default font (+ math font) than Computer Modern for most use cases
    \usepackage{mathpazo}

    % Basic figure setup, for now with no caption control since it's done
    % automatically by Pandoc (which extracts ![](path) syntax from Markdown).
    \usepackage{graphicx}
    % We will generate all images so they have a width \maxwidth. This means
    % that they will get their normal width if they fit onto the page, but
    % are scaled down if they would overflow the margins.
    \makeatletter
    \def\maxwidth{\ifdim\Gin@nat@width>\linewidth\linewidth
    \else\Gin@nat@width\fi}
    \makeatother
    \let\Oldincludegraphics\includegraphics
    % Set max figure width to be 80% of text width, for now hardcoded.
    \renewcommand{\includegraphics}[1]{\Oldincludegraphics[width=.8\maxwidth]{#1}}
    % Ensure that by default, figures have no caption (until we provide a
    % proper Figure object with a Caption API and a way to capture that
    % in the conversion process - todo).
    \usepackage{caption}
    \DeclareCaptionLabelFormat{nolabel}{}
    \captionsetup{labelformat=nolabel}

    \usepackage{adjustbox} % Used to constrain images to a maximum size 
    \usepackage{xcolor} % Allow colors to be defined
    \usepackage{enumerate} % Needed for markdown enumerations to work
    \usepackage{geometry} % Used to adjust the document margins
    \usepackage{amsmath} % Equations
    \usepackage{amssymb} % Equations
    \usepackage{textcomp} % defines textquotesingle
    % Hack from http://tex.stackexchange.com/a/47451/13684:
    \AtBeginDocument{%
        \def\PYZsq{\textquotesingle}% Upright quotes in Pygmentized code
    }
    \usepackage{upquote} % Upright quotes for verbatim code
    \usepackage{eurosym} % defines \euro
    \usepackage[mathletters]{ucs} % Extended unicode (utf-8) support
    \usepackage[utf8x]{inputenc} % Allow utf-8 characters in the tex document
    \usepackage{fancyvrb} % verbatim replacement that allows latex
    \usepackage{grffile} % extends the file name processing of package graphics 
                         % to support a larger range 
    % The hyperref package gives us a pdf with properly built
    % internal navigation ('pdf bookmarks' for the table of contents,
    % internal cross-reference links, web links for URLs, etc.)
    \usepackage{hyperref}
    \usepackage{longtable} % longtable support required by pandoc >1.10
    \usepackage{booktabs}  % table support for pandoc > 1.12.2
    \usepackage[inline]{enumitem} % IRkernel/repr support (it uses the enumerate* environment)
    \usepackage[normalem]{ulem} % ulem is needed to support strikethroughs (\sout)
                                % normalem makes italics be italics, not underlines
    

    
    
    % Colors for the hyperref package
    \definecolor{urlcolor}{rgb}{0,.145,.698}
    \definecolor{linkcolor}{rgb}{.71,0.21,0.01}
    \definecolor{citecolor}{rgb}{.12,.54,.11}

    % ANSI colors
    \definecolor{ansi-black}{HTML}{3E424D}
    \definecolor{ansi-black-intense}{HTML}{282C36}
    \definecolor{ansi-red}{HTML}{E75C58}
    \definecolor{ansi-red-intense}{HTML}{B22B31}
    \definecolor{ansi-green}{HTML}{00A250}
    \definecolor{ansi-green-intense}{HTML}{007427}
    \definecolor{ansi-yellow}{HTML}{DDB62B}
    \definecolor{ansi-yellow-intense}{HTML}{B27D12}
    \definecolor{ansi-blue}{HTML}{208FFB}
    \definecolor{ansi-blue-intense}{HTML}{0065CA}
    \definecolor{ansi-magenta}{HTML}{D160C4}
    \definecolor{ansi-magenta-intense}{HTML}{A03196}
    \definecolor{ansi-cyan}{HTML}{60C6C8}
    \definecolor{ansi-cyan-intense}{HTML}{258F8F}
    \definecolor{ansi-white}{HTML}{C5C1B4}
    \definecolor{ansi-white-intense}{HTML}{A1A6B2}

    % commands and environments needed by pandoc snippets
    % extracted from the output of `pandoc -s`
    \providecommand{\tightlist}{%
      \setlength{\itemsep}{0pt}\setlength{\parskip}{0pt}}
    \DefineVerbatimEnvironment{Highlighting}{Verbatim}{commandchars=\\\{\}}
    % Add ',fontsize=\small' for more characters per line
    \newenvironment{Shaded}{}{}
    \newcommand{\KeywordTok}[1]{\textcolor[rgb]{0.00,0.44,0.13}{\textbf{{#1}}}}
    \newcommand{\DataTypeTok}[1]{\textcolor[rgb]{0.56,0.13,0.00}{{#1}}}
    \newcommand{\DecValTok}[1]{\textcolor[rgb]{0.25,0.63,0.44}{{#1}}}
    \newcommand{\BaseNTok}[1]{\textcolor[rgb]{0.25,0.63,0.44}{{#1}}}
    \newcommand{\FloatTok}[1]{\textcolor[rgb]{0.25,0.63,0.44}{{#1}}}
    \newcommand{\CharTok}[1]{\textcolor[rgb]{0.25,0.44,0.63}{{#1}}}
    \newcommand{\StringTok}[1]{\textcolor[rgb]{0.25,0.44,0.63}{{#1}}}
    \newcommand{\CommentTok}[1]{\textcolor[rgb]{0.38,0.63,0.69}{\textit{{#1}}}}
    \newcommand{\OtherTok}[1]{\textcolor[rgb]{0.00,0.44,0.13}{{#1}}}
    \newcommand{\AlertTok}[1]{\textcolor[rgb]{1.00,0.00,0.00}{\textbf{{#1}}}}
    \newcommand{\FunctionTok}[1]{\textcolor[rgb]{0.02,0.16,0.49}{{#1}}}
    \newcommand{\RegionMarkerTok}[1]{{#1}}
    \newcommand{\ErrorTok}[1]{\textcolor[rgb]{1.00,0.00,0.00}{\textbf{{#1}}}}
    \newcommand{\NormalTok}[1]{{#1}}
    
    % Additional commands for more recent versions of Pandoc
    \newcommand{\ConstantTok}[1]{\textcolor[rgb]{0.53,0.00,0.00}{{#1}}}
    \newcommand{\SpecialCharTok}[1]{\textcolor[rgb]{0.25,0.44,0.63}{{#1}}}
    \newcommand{\VerbatimStringTok}[1]{\textcolor[rgb]{0.25,0.44,0.63}{{#1}}}
    \newcommand{\SpecialStringTok}[1]{\textcolor[rgb]{0.73,0.40,0.53}{{#1}}}
    \newcommand{\ImportTok}[1]{{#1}}
    \newcommand{\DocumentationTok}[1]{\textcolor[rgb]{0.73,0.13,0.13}{\textit{{#1}}}}
    \newcommand{\AnnotationTok}[1]{\textcolor[rgb]{0.38,0.63,0.69}{\textbf{\textit{{#1}}}}}
    \newcommand{\CommentVarTok}[1]{\textcolor[rgb]{0.38,0.63,0.69}{\textbf{\textit{{#1}}}}}
    \newcommand{\VariableTok}[1]{\textcolor[rgb]{0.10,0.09,0.49}{{#1}}}
    \newcommand{\ControlFlowTok}[1]{\textcolor[rgb]{0.00,0.44,0.13}{\textbf{{#1}}}}
    \newcommand{\OperatorTok}[1]{\textcolor[rgb]{0.40,0.40,0.40}{{#1}}}
    \newcommand{\BuiltInTok}[1]{{#1}}
    \newcommand{\ExtensionTok}[1]{{#1}}
    \newcommand{\PreprocessorTok}[1]{\textcolor[rgb]{0.74,0.48,0.00}{{#1}}}
    \newcommand{\AttributeTok}[1]{\textcolor[rgb]{0.49,0.56,0.16}{{#1}}}
    \newcommand{\InformationTok}[1]{\textcolor[rgb]{0.38,0.63,0.69}{\textbf{\textit{{#1}}}}}
    \newcommand{\WarningTok}[1]{\textcolor[rgb]{0.38,0.63,0.69}{\textbf{\textit{{#1}}}}}
    
    
    % Define a nice break command that doesn't care if a line doesn't already
    % exist.
    \def\br{\hspace*{\fill} \\* }
    % Math Jax compatability definitions
    \def\gt{>}
    \def\lt{<}
    % Document parameters
    \title{cat dog classifier CNN model}
    
    
    

    % Pygments definitions
    
\makeatletter
\def\PY@reset{\let\PY@it=\relax \let\PY@bf=\relax%
    \let\PY@ul=\relax \let\PY@tc=\relax%
    \let\PY@bc=\relax \let\PY@ff=\relax}
\def\PY@tok#1{\csname PY@tok@#1\endcsname}
\def\PY@toks#1+{\ifx\relax#1\empty\else%
    \PY@tok{#1}\expandafter\PY@toks\fi}
\def\PY@do#1{\PY@bc{\PY@tc{\PY@ul{%
    \PY@it{\PY@bf{\PY@ff{#1}}}}}}}
\def\PY#1#2{\PY@reset\PY@toks#1+\relax+\PY@do{#2}}

\expandafter\def\csname PY@tok@nf\endcsname{\def\PY@tc##1{\textcolor[rgb]{0.00,0.00,1.00}{##1}}}
\expandafter\def\csname PY@tok@mi\endcsname{\def\PY@tc##1{\textcolor[rgb]{0.40,0.40,0.40}{##1}}}
\expandafter\def\csname PY@tok@mo\endcsname{\def\PY@tc##1{\textcolor[rgb]{0.40,0.40,0.40}{##1}}}
\expandafter\def\csname PY@tok@w\endcsname{\def\PY@tc##1{\textcolor[rgb]{0.73,0.73,0.73}{##1}}}
\expandafter\def\csname PY@tok@no\endcsname{\def\PY@tc##1{\textcolor[rgb]{0.53,0.00,0.00}{##1}}}
\expandafter\def\csname PY@tok@sc\endcsname{\def\PY@tc##1{\textcolor[rgb]{0.73,0.13,0.13}{##1}}}
\expandafter\def\csname PY@tok@cpf\endcsname{\let\PY@it=\textit\def\PY@tc##1{\textcolor[rgb]{0.25,0.50,0.50}{##1}}}
\expandafter\def\csname PY@tok@mh\endcsname{\def\PY@tc##1{\textcolor[rgb]{0.40,0.40,0.40}{##1}}}
\expandafter\def\csname PY@tok@mf\endcsname{\def\PY@tc##1{\textcolor[rgb]{0.40,0.40,0.40}{##1}}}
\expandafter\def\csname PY@tok@kr\endcsname{\let\PY@bf=\textbf\def\PY@tc##1{\textcolor[rgb]{0.00,0.50,0.00}{##1}}}
\expandafter\def\csname PY@tok@gr\endcsname{\def\PY@tc##1{\textcolor[rgb]{1.00,0.00,0.00}{##1}}}
\expandafter\def\csname PY@tok@nt\endcsname{\let\PY@bf=\textbf\def\PY@tc##1{\textcolor[rgb]{0.00,0.50,0.00}{##1}}}
\expandafter\def\csname PY@tok@s2\endcsname{\def\PY@tc##1{\textcolor[rgb]{0.73,0.13,0.13}{##1}}}
\expandafter\def\csname PY@tok@gt\endcsname{\def\PY@tc##1{\textcolor[rgb]{0.00,0.27,0.87}{##1}}}
\expandafter\def\csname PY@tok@c1\endcsname{\let\PY@it=\textit\def\PY@tc##1{\textcolor[rgb]{0.25,0.50,0.50}{##1}}}
\expandafter\def\csname PY@tok@gu\endcsname{\let\PY@bf=\textbf\def\PY@tc##1{\textcolor[rgb]{0.50,0.00,0.50}{##1}}}
\expandafter\def\csname PY@tok@kt\endcsname{\def\PY@tc##1{\textcolor[rgb]{0.69,0.00,0.25}{##1}}}
\expandafter\def\csname PY@tok@k\endcsname{\let\PY@bf=\textbf\def\PY@tc##1{\textcolor[rgb]{0.00,0.50,0.00}{##1}}}
\expandafter\def\csname PY@tok@nv\endcsname{\def\PY@tc##1{\textcolor[rgb]{0.10,0.09,0.49}{##1}}}
\expandafter\def\csname PY@tok@mb\endcsname{\def\PY@tc##1{\textcolor[rgb]{0.40,0.40,0.40}{##1}}}
\expandafter\def\csname PY@tok@cp\endcsname{\def\PY@tc##1{\textcolor[rgb]{0.74,0.48,0.00}{##1}}}
\expandafter\def\csname PY@tok@c\endcsname{\let\PY@it=\textit\def\PY@tc##1{\textcolor[rgb]{0.25,0.50,0.50}{##1}}}
\expandafter\def\csname PY@tok@m\endcsname{\def\PY@tc##1{\textcolor[rgb]{0.40,0.40,0.40}{##1}}}
\expandafter\def\csname PY@tok@go\endcsname{\def\PY@tc##1{\textcolor[rgb]{0.53,0.53,0.53}{##1}}}
\expandafter\def\csname PY@tok@s1\endcsname{\def\PY@tc##1{\textcolor[rgb]{0.73,0.13,0.13}{##1}}}
\expandafter\def\csname PY@tok@vi\endcsname{\def\PY@tc##1{\textcolor[rgb]{0.10,0.09,0.49}{##1}}}
\expandafter\def\csname PY@tok@na\endcsname{\def\PY@tc##1{\textcolor[rgb]{0.49,0.56,0.16}{##1}}}
\expandafter\def\csname PY@tok@si\endcsname{\let\PY@bf=\textbf\def\PY@tc##1{\textcolor[rgb]{0.73,0.40,0.53}{##1}}}
\expandafter\def\csname PY@tok@ow\endcsname{\let\PY@bf=\textbf\def\PY@tc##1{\textcolor[rgb]{0.67,0.13,1.00}{##1}}}
\expandafter\def\csname PY@tok@sd\endcsname{\let\PY@it=\textit\def\PY@tc##1{\textcolor[rgb]{0.73,0.13,0.13}{##1}}}
\expandafter\def\csname PY@tok@nd\endcsname{\def\PY@tc##1{\textcolor[rgb]{0.67,0.13,1.00}{##1}}}
\expandafter\def\csname PY@tok@gi\endcsname{\def\PY@tc##1{\textcolor[rgb]{0.00,0.63,0.00}{##1}}}
\expandafter\def\csname PY@tok@kp\endcsname{\def\PY@tc##1{\textcolor[rgb]{0.00,0.50,0.00}{##1}}}
\expandafter\def\csname PY@tok@fm\endcsname{\def\PY@tc##1{\textcolor[rgb]{0.00,0.00,1.00}{##1}}}
\expandafter\def\csname PY@tok@gd\endcsname{\def\PY@tc##1{\textcolor[rgb]{0.63,0.00,0.00}{##1}}}
\expandafter\def\csname PY@tok@sb\endcsname{\def\PY@tc##1{\textcolor[rgb]{0.73,0.13,0.13}{##1}}}
\expandafter\def\csname PY@tok@o\endcsname{\def\PY@tc##1{\textcolor[rgb]{0.40,0.40,0.40}{##1}}}
\expandafter\def\csname PY@tok@vc\endcsname{\def\PY@tc##1{\textcolor[rgb]{0.10,0.09,0.49}{##1}}}
\expandafter\def\csname PY@tok@sa\endcsname{\def\PY@tc##1{\textcolor[rgb]{0.73,0.13,0.13}{##1}}}
\expandafter\def\csname PY@tok@ss\endcsname{\def\PY@tc##1{\textcolor[rgb]{0.10,0.09,0.49}{##1}}}
\expandafter\def\csname PY@tok@s\endcsname{\def\PY@tc##1{\textcolor[rgb]{0.73,0.13,0.13}{##1}}}
\expandafter\def\csname PY@tok@nn\endcsname{\let\PY@bf=\textbf\def\PY@tc##1{\textcolor[rgb]{0.00,0.00,1.00}{##1}}}
\expandafter\def\csname PY@tok@vg\endcsname{\def\PY@tc##1{\textcolor[rgb]{0.10,0.09,0.49}{##1}}}
\expandafter\def\csname PY@tok@cs\endcsname{\let\PY@it=\textit\def\PY@tc##1{\textcolor[rgb]{0.25,0.50,0.50}{##1}}}
\expandafter\def\csname PY@tok@vm\endcsname{\def\PY@tc##1{\textcolor[rgb]{0.10,0.09,0.49}{##1}}}
\expandafter\def\csname PY@tok@nl\endcsname{\def\PY@tc##1{\textcolor[rgb]{0.63,0.63,0.00}{##1}}}
\expandafter\def\csname PY@tok@ge\endcsname{\let\PY@it=\textit}
\expandafter\def\csname PY@tok@gs\endcsname{\let\PY@bf=\textbf}
\expandafter\def\csname PY@tok@nc\endcsname{\let\PY@bf=\textbf\def\PY@tc##1{\textcolor[rgb]{0.00,0.00,1.00}{##1}}}
\expandafter\def\csname PY@tok@kn\endcsname{\let\PY@bf=\textbf\def\PY@tc##1{\textcolor[rgb]{0.00,0.50,0.00}{##1}}}
\expandafter\def\csname PY@tok@err\endcsname{\def\PY@bc##1{\setlength{\fboxsep}{0pt}\fcolorbox[rgb]{1.00,0.00,0.00}{1,1,1}{\strut ##1}}}
\expandafter\def\csname PY@tok@kc\endcsname{\let\PY@bf=\textbf\def\PY@tc##1{\textcolor[rgb]{0.00,0.50,0.00}{##1}}}
\expandafter\def\csname PY@tok@dl\endcsname{\def\PY@tc##1{\textcolor[rgb]{0.73,0.13,0.13}{##1}}}
\expandafter\def\csname PY@tok@cm\endcsname{\let\PY@it=\textit\def\PY@tc##1{\textcolor[rgb]{0.25,0.50,0.50}{##1}}}
\expandafter\def\csname PY@tok@nb\endcsname{\def\PY@tc##1{\textcolor[rgb]{0.00,0.50,0.00}{##1}}}
\expandafter\def\csname PY@tok@bp\endcsname{\def\PY@tc##1{\textcolor[rgb]{0.00,0.50,0.00}{##1}}}
\expandafter\def\csname PY@tok@ch\endcsname{\let\PY@it=\textit\def\PY@tc##1{\textcolor[rgb]{0.25,0.50,0.50}{##1}}}
\expandafter\def\csname PY@tok@sr\endcsname{\def\PY@tc##1{\textcolor[rgb]{0.73,0.40,0.53}{##1}}}
\expandafter\def\csname PY@tok@se\endcsname{\let\PY@bf=\textbf\def\PY@tc##1{\textcolor[rgb]{0.73,0.40,0.13}{##1}}}
\expandafter\def\csname PY@tok@ni\endcsname{\let\PY@bf=\textbf\def\PY@tc##1{\textcolor[rgb]{0.60,0.60,0.60}{##1}}}
\expandafter\def\csname PY@tok@ne\endcsname{\let\PY@bf=\textbf\def\PY@tc##1{\textcolor[rgb]{0.82,0.25,0.23}{##1}}}
\expandafter\def\csname PY@tok@kd\endcsname{\let\PY@bf=\textbf\def\PY@tc##1{\textcolor[rgb]{0.00,0.50,0.00}{##1}}}
\expandafter\def\csname PY@tok@gh\endcsname{\let\PY@bf=\textbf\def\PY@tc##1{\textcolor[rgb]{0.00,0.00,0.50}{##1}}}
\expandafter\def\csname PY@tok@il\endcsname{\def\PY@tc##1{\textcolor[rgb]{0.40,0.40,0.40}{##1}}}
\expandafter\def\csname PY@tok@sh\endcsname{\def\PY@tc##1{\textcolor[rgb]{0.73,0.13,0.13}{##1}}}
\expandafter\def\csname PY@tok@gp\endcsname{\let\PY@bf=\textbf\def\PY@tc##1{\textcolor[rgb]{0.00,0.00,0.50}{##1}}}
\expandafter\def\csname PY@tok@sx\endcsname{\def\PY@tc##1{\textcolor[rgb]{0.00,0.50,0.00}{##1}}}

\def\PYZbs{\char`\\}
\def\PYZus{\char`\_}
\def\PYZob{\char`\{}
\def\PYZcb{\char`\}}
\def\PYZca{\char`\^}
\def\PYZam{\char`\&}
\def\PYZlt{\char`\<}
\def\PYZgt{\char`\>}
\def\PYZsh{\char`\#}
\def\PYZpc{\char`\%}
\def\PYZdl{\char`\$}
\def\PYZhy{\char`\-}
\def\PYZsq{\char`\'}
\def\PYZdq{\char`\"}
\def\PYZti{\char`\~}
% for compatibility with earlier versions
\def\PYZat{@}
\def\PYZlb{[}
\def\PYZrb{]}
\makeatother


    % Exact colors from NB
    \definecolor{incolor}{rgb}{0.0, 0.0, 0.5}
    \definecolor{outcolor}{rgb}{0.545, 0.0, 0.0}



    
    % Prevent overflowing lines due to hard-to-break entities
    \sloppy 
    % Setup hyperref package
    \hypersetup{
      breaklinks=true,  % so long urls are correctly broken across lines
      colorlinks=true,
      urlcolor=urlcolor,
      linkcolor=linkcolor,
      citecolor=citecolor,
      }
    % Slightly bigger margins than the latex defaults
    
    \geometry{verbose,tmargin=1in,bmargin=1in,lmargin=1in,rmargin=1in}
    
    

    \begin{document}
    
    
    \maketitle
    
    

    
    \section{TensorFlow Convolutional Neural Network for Image
Classification}\label{tensorflow-convolutional-neural-network-for-image-classification}

    \begin{Verbatim}[commandchars=\\\{\}]
{\color{incolor}In [{\color{incolor}1}]:} \PY{k+kn}{import} \PY{n+nn}{time}
        \PY{k+kn}{import} \PY{n+nn}{math}
        \PY{k+kn}{import} \PY{n+nn}{random}
        
        \PY{k+kn}{import} \PY{n+nn}{pandas} \PY{k}{as} \PY{n+nn}{pd}
        \PY{k+kn}{import} \PY{n+nn}{numpy} \PY{k}{as} \PY{n+nn}{np}
        \PY{k+kn}{import} \PY{n+nn}{matplotlib}\PY{n+nn}{.}\PY{n+nn}{pyplot} \PY{k}{as} \PY{n+nn}{plt}
        \PY{k+kn}{import} \PY{n+nn}{tensorflow} \PY{k}{as} \PY{n+nn}{tf}
        \PY{k+kn}{import} \PY{n+nn}{dataset}
        \PY{k+kn}{import} \PY{n+nn}{cv2}
        
        \PY{k+kn}{from} \PY{n+nn}{sklearn}\PY{n+nn}{.}\PY{n+nn}{metrics} \PY{k}{import} \PY{n}{confusion\PYZus{}matrix}
        \PY{k+kn}{from} \PY{n+nn}{datetime} \PY{k}{import} \PY{n}{timedelta}
        
        \PY{o}{\PYZpc{}}\PY{k}{matplotlib} inline
\end{Verbatim}


    \begin{Verbatim}[commandchars=\\\{\}]
C:\textbackslash{}Users\textbackslash{}ameypawar\textbackslash{}Anaconda2\textbackslash{}envs\textbackslash{}python3\textbackslash{}lib\textbackslash{}site-packages\textbackslash{}h5py\textbackslash{}\_\_init\_\_.py:36: FutureWarning: Conversion of the second argument of issubdtype from `float` to `np.floating` is deprecated. In future, it will be treated as `np.float64 == np.dtype(float).type`.
  from .\_conv import register\_converters as \_register\_converters

    \end{Verbatim}

    \subsection{Configuration and
Hyperparameters}\label{configuration-and-hyperparameters}

    \begin{Verbatim}[commandchars=\\\{\}]
{\color{incolor}In [{\color{incolor}2}]:} \PY{c+c1}{\PYZsh{} Convolutional Layer 1.}
        \PY{n}{filter\PYZus{}size1} \PY{o}{=} \PY{l+m+mi}{3} 
        \PY{n}{num\PYZus{}filters1} \PY{o}{=} \PY{l+m+mi}{32}
        
        \PY{c+c1}{\PYZsh{} Convolutional Layer 2.}
        \PY{n}{filter\PYZus{}size2} \PY{o}{=} \PY{l+m+mi}{3}
        \PY{n}{num\PYZus{}filters2} \PY{o}{=} \PY{l+m+mi}{32}
        
        \PY{c+c1}{\PYZsh{} Convolutional Layer 3.}
        \PY{n}{filter\PYZus{}size3} \PY{o}{=} \PY{l+m+mi}{3}
        \PY{n}{num\PYZus{}filters3} \PY{o}{=} \PY{l+m+mi}{64}
        
        \PY{c+c1}{\PYZsh{} Fully\PYZhy{}connected layer.}
        \PY{n}{fc\PYZus{}size} \PY{o}{=} \PY{l+m+mi}{128}             \PY{c+c1}{\PYZsh{} Number of neurons in fully\PYZhy{}connected layer.}
        
        \PY{c+c1}{\PYZsh{} Number of color channels for the images: 1 channel for gray\PYZhy{}scale.}
        \PY{n}{num\PYZus{}channels} \PY{o}{=} \PY{l+m+mi}{3}
        
        \PY{c+c1}{\PYZsh{} image dimensions (only squares for now)}
        \PY{n}{img\PYZus{}size} \PY{o}{=} \PY{l+m+mi}{128}
        
        \PY{c+c1}{\PYZsh{} Size of image when flattened to a single dimension}
        \PY{n}{img\PYZus{}size\PYZus{}flat} \PY{o}{=} \PY{n}{img\PYZus{}size} \PY{o}{*} \PY{n}{img\PYZus{}size} \PY{o}{*} \PY{n}{num\PYZus{}channels}
        
        \PY{c+c1}{\PYZsh{} Tuple with height and width of images used to reshape arrays.}
        \PY{n}{img\PYZus{}shape} \PY{o}{=} \PY{p}{(}\PY{n}{img\PYZus{}size}\PY{p}{,} \PY{n}{img\PYZus{}size}\PY{p}{)}
        
        \PY{c+c1}{\PYZsh{} class info}
        \PY{n}{classes} \PY{o}{=} \PY{p}{[}\PY{l+s+s1}{\PYZsq{}}\PY{l+s+s1}{dogs}\PY{l+s+s1}{\PYZsq{}}\PY{p}{,} \PY{l+s+s1}{\PYZsq{}}\PY{l+s+s1}{cats}\PY{l+s+s1}{\PYZsq{}}\PY{p}{]}
        \PY{n}{num\PYZus{}classes} \PY{o}{=} \PY{n+nb}{len}\PY{p}{(}\PY{n}{classes}\PY{p}{)}
        
        \PY{c+c1}{\PYZsh{} batch size}
        \PY{n}{batch\PYZus{}size} \PY{o}{=} \PY{l+m+mi}{32}
        
        \PY{c+c1}{\PYZsh{} validation split}
        \PY{n}{validation\PYZus{}size} \PY{o}{=} \PY{o}{.}\PY{l+m+mi}{16}
        
        \PY{c+c1}{\PYZsh{} how long to wait after validation loss stops improving before terminating training}
        \PY{n}{early\PYZus{}stopping} \PY{o}{=} \PY{k+kc}{None}  \PY{c+c1}{\PYZsh{} use None if you don\PYZsq{}t want to implement early stoping}
        
        \PY{n}{train\PYZus{}path} \PY{o}{=} \PY{l+s+s1}{\PYZsq{}}\PY{l+s+s1}{data/train/train/}\PY{l+s+s1}{\PYZsq{}}
        \PY{n}{test\PYZus{}path} \PY{o}{=} \PY{l+s+s1}{\PYZsq{}}\PY{l+s+s1}{data/test/test/}\PY{l+s+s1}{\PYZsq{}}
        \PY{n}{checkpoint\PYZus{}dir} \PY{o}{=} \PY{l+s+s2}{\PYZdq{}}\PY{l+s+s2}{models/}\PY{l+s+s2}{\PYZdq{}}
\end{Verbatim}


    \subsection{Load Data}\label{load-data}

    \begin{Verbatim}[commandchars=\\\{\}]
{\color{incolor}In [{\color{incolor}3}]:} \PY{n}{data} \PY{o}{=} \PY{n}{dataset}\PY{o}{.}\PY{n}{read\PYZus{}train\PYZus{}sets}\PY{p}{(}\PY{n}{train\PYZus{}path}\PY{p}{,} \PY{n}{img\PYZus{}size}\PY{p}{,} \PY{n}{classes}\PY{p}{,} \PY{n}{validation\PYZus{}size}\PY{o}{=}\PY{n}{validation\PYZus{}size}\PY{p}{)}
        \PY{n}{test\PYZus{}images}\PY{p}{,} \PY{n}{test\PYZus{}ids} \PY{o}{=} \PY{n}{dataset}\PY{o}{.}\PY{n}{read\PYZus{}test\PYZus{}set}\PY{p}{(}\PY{n}{test\PYZus{}path}\PY{p}{,} \PY{n}{img\PYZus{}size}\PY{p}{)}
\end{Verbatim}


    \begin{Verbatim}[commandchars=\\\{\}]
Reading training images
Loading dogs files (Index: 0)
Loading cats files (Index: 1)
Reading test images

    \end{Verbatim}

    \begin{Verbatim}[commandchars=\\\{\}]
{\color{incolor}In [{\color{incolor}4}]:} \PY{n+nb}{print}\PY{p}{(}\PY{l+s+s2}{\PYZdq{}}\PY{l+s+s2}{Size of:}\PY{l+s+s2}{\PYZdq{}}\PY{p}{)}
        \PY{n+nb}{print}\PY{p}{(}\PY{l+s+s2}{\PYZdq{}}\PY{l+s+s2}{\PYZhy{} Training\PYZhy{}set:}\PY{l+s+se}{\PYZbs{}t}\PY{l+s+se}{\PYZbs{}t}\PY{l+s+si}{\PYZob{}\PYZcb{}}\PY{l+s+s2}{\PYZdq{}}\PY{o}{.}\PY{n}{format}\PY{p}{(}\PY{n+nb}{len}\PY{p}{(}\PY{n}{data}\PY{o}{.}\PY{n}{train}\PY{o}{.}\PY{n}{labels}\PY{p}{)}\PY{p}{)}\PY{p}{)}
        \PY{n+nb}{print}\PY{p}{(}\PY{l+s+s2}{\PYZdq{}}\PY{l+s+s2}{\PYZhy{} Test\PYZhy{}set:}\PY{l+s+se}{\PYZbs{}t}\PY{l+s+se}{\PYZbs{}t}\PY{l+s+si}{\PYZob{}\PYZcb{}}\PY{l+s+s2}{\PYZdq{}}\PY{o}{.}\PY{n}{format}\PY{p}{(}\PY{n+nb}{len}\PY{p}{(}\PY{n}{test\PYZus{}images}\PY{p}{)}\PY{p}{)}\PY{p}{)}
        \PY{n+nb}{print}\PY{p}{(}\PY{l+s+s2}{\PYZdq{}}\PY{l+s+s2}{\PYZhy{} Validation\PYZhy{}set:}\PY{l+s+se}{\PYZbs{}t}\PY{l+s+si}{\PYZob{}\PYZcb{}}\PY{l+s+s2}{\PYZdq{}}\PY{o}{.}\PY{n}{format}\PY{p}{(}\PY{n+nb}{len}\PY{p}{(}\PY{n}{data}\PY{o}{.}\PY{n}{valid}\PY{o}{.}\PY{n}{labels}\PY{p}{)}\PY{p}{)}\PY{p}{)}
\end{Verbatim}


    \begin{Verbatim}[commandchars=\\\{\}]
Size of:
- Training-set:		21000
- Test-set:		12500
- Validation-set:	4000

    \end{Verbatim}

    \subsubsection{Helper-function for plotting
images}\label{helper-function-for-plotting-images}

    Function used to plot 9 images in a 3x3 grid (or fewer, depending on how
many images are passed), and writing the true and predicted classes
below each image.

    \begin{Verbatim}[commandchars=\\\{\}]
{\color{incolor}In [{\color{incolor}5}]:} \PY{k}{def} \PY{n+nf}{plot\PYZus{}images}\PY{p}{(}\PY{n}{images}\PY{p}{,} \PY{n}{cls\PYZus{}true}\PY{p}{,} \PY{n}{cls\PYZus{}pred}\PY{o}{=}\PY{k+kc}{None}\PY{p}{)}\PY{p}{:}
            
            \PY{k}{if} \PY{n+nb}{len}\PY{p}{(}\PY{n}{images}\PY{p}{)} \PY{o}{==} \PY{l+m+mi}{0}\PY{p}{:}
                \PY{n+nb}{print}\PY{p}{(}\PY{l+s+s2}{\PYZdq{}}\PY{l+s+s2}{no images to show}\PY{l+s+s2}{\PYZdq{}}\PY{p}{)}
                \PY{k}{return} 
            \PY{k}{else}\PY{p}{:}
                \PY{n}{random\PYZus{}indices} \PY{o}{=} \PY{n}{random}\PY{o}{.}\PY{n}{sample}\PY{p}{(}\PY{n+nb}{range}\PY{p}{(}\PY{n+nb}{len}\PY{p}{(}\PY{n}{images}\PY{p}{)}\PY{p}{)}\PY{p}{,} \PY{n+nb}{min}\PY{p}{(}\PY{n+nb}{len}\PY{p}{(}\PY{n}{images}\PY{p}{)}\PY{p}{,} \PY{l+m+mi}{9}\PY{p}{)}\PY{p}{)}
                
                
            \PY{n}{images}\PY{p}{,} \PY{n}{cls\PYZus{}true}  \PY{o}{=} \PY{n+nb}{zip}\PY{p}{(}\PY{o}{*}\PY{p}{[}\PY{p}{(}\PY{n}{images}\PY{p}{[}\PY{n}{i}\PY{p}{]}\PY{p}{,} \PY{n}{cls\PYZus{}true}\PY{p}{[}\PY{n}{i}\PY{p}{]}\PY{p}{)} \PY{k}{for} \PY{n}{i} \PY{o+ow}{in} \PY{n}{random\PYZus{}indices}\PY{p}{]}\PY{p}{)}
            
            \PY{c+c1}{\PYZsh{} Create figure with 3x3 sub\PYZhy{}plots.}
            \PY{n}{fig}\PY{p}{,} \PY{n}{axes} \PY{o}{=} \PY{n}{plt}\PY{o}{.}\PY{n}{subplots}\PY{p}{(}\PY{l+m+mi}{3}\PY{p}{,} \PY{l+m+mi}{3}\PY{p}{)}
            \PY{n}{fig}\PY{o}{.}\PY{n}{subplots\PYZus{}adjust}\PY{p}{(}\PY{n}{hspace}\PY{o}{=}\PY{l+m+mf}{0.3}\PY{p}{,} \PY{n}{wspace}\PY{o}{=}\PY{l+m+mf}{0.3}\PY{p}{)}
        
            \PY{k}{for} \PY{n}{i}\PY{p}{,} \PY{n}{ax} \PY{o+ow}{in} \PY{n+nb}{enumerate}\PY{p}{(}\PY{n}{axes}\PY{o}{.}\PY{n}{flat}\PY{p}{)}\PY{p}{:}
                \PY{c+c1}{\PYZsh{} Plot image.}
                \PY{n}{ax}\PY{o}{.}\PY{n}{imshow}\PY{p}{(}\PY{n}{images}\PY{p}{[}\PY{n}{i}\PY{p}{]}\PY{o}{.}\PY{n}{reshape}\PY{p}{(}\PY{n}{img\PYZus{}size}\PY{p}{,} \PY{n}{img\PYZus{}size}\PY{p}{,} \PY{n}{num\PYZus{}channels}\PY{p}{)}\PY{p}{)}
        
                \PY{c+c1}{\PYZsh{} Show true and predicted classes.}
                \PY{k}{if} \PY{n}{cls\PYZus{}pred} \PY{o+ow}{is} \PY{k+kc}{None}\PY{p}{:}
                    \PY{n}{xlabel} \PY{o}{=} \PY{l+s+s2}{\PYZdq{}}\PY{l+s+s2}{True: }\PY{l+s+si}{\PYZob{}0\PYZcb{}}\PY{l+s+s2}{\PYZdq{}}\PY{o}{.}\PY{n}{format}\PY{p}{(}\PY{n}{cls\PYZus{}true}\PY{p}{[}\PY{n}{i}\PY{p}{]}\PY{p}{)}
                \PY{k}{else}\PY{p}{:}
                    \PY{n}{xlabel} \PY{o}{=} \PY{l+s+s2}{\PYZdq{}}\PY{l+s+s2}{True: }\PY{l+s+si}{\PYZob{}0\PYZcb{}}\PY{l+s+s2}{, Pred: }\PY{l+s+si}{\PYZob{}1\PYZcb{}}\PY{l+s+s2}{\PYZdq{}}\PY{o}{.}\PY{n}{format}\PY{p}{(}\PY{n}{cls\PYZus{}true}\PY{p}{[}\PY{n}{i}\PY{p}{]}\PY{p}{,} \PY{n}{cls\PYZus{}pred}\PY{p}{[}\PY{n}{i}\PY{p}{]}\PY{p}{)}
        
                \PY{c+c1}{\PYZsh{} Show the classes as the label on the x\PYZhy{}axis.}
                \PY{n}{ax}\PY{o}{.}\PY{n}{set\PYZus{}xlabel}\PY{p}{(}\PY{n}{xlabel}\PY{p}{)}
                
                \PY{c+c1}{\PYZsh{} Remove ticks from the plot.}
                \PY{n}{ax}\PY{o}{.}\PY{n}{set\PYZus{}xticks}\PY{p}{(}\PY{p}{[}\PY{p}{]}\PY{p}{)}
                \PY{n}{ax}\PY{o}{.}\PY{n}{set\PYZus{}yticks}\PY{p}{(}\PY{p}{[}\PY{p}{]}\PY{p}{)}
            
            \PY{c+c1}{\PYZsh{} Ensure the plot is shown correctly with multiple plots}
            \PY{c+c1}{\PYZsh{} in a single Notebook cell.}
            \PY{n}{plt}\PY{o}{.}\PY{n}{show}\PY{p}{(}\PY{p}{)}
\end{Verbatim}


    \subsubsection{Plot a few images to see if data is
correct}\label{plot-a-few-images-to-see-if-data-is-correct}

    \begin{Verbatim}[commandchars=\\\{\}]
{\color{incolor}In [{\color{incolor}6}]:} \PY{c+c1}{\PYZsh{} Get some random images and their labels from the train set.}
        
        \PY{n}{images}\PY{p}{,} \PY{n}{cls\PYZus{}true}  \PY{o}{=} \PY{n}{data}\PY{o}{.}\PY{n}{train}\PY{o}{.}\PY{n}{images}\PY{p}{,} \PY{n}{data}\PY{o}{.}\PY{n}{train}\PY{o}{.}\PY{n}{cls}
        
        \PY{c+c1}{\PYZsh{} Plot the images and labels using our helper\PYZhy{}function above.}
        \PY{n}{plot\PYZus{}images}\PY{p}{(}\PY{n}{images}\PY{o}{=}\PY{n}{images}\PY{p}{,} \PY{n}{cls\PYZus{}true}\PY{o}{=}\PY{n}{cls\PYZus{}true}\PY{p}{)}
\end{Verbatim}


    \begin{center}
    \adjustimage{max size={0.9\linewidth}{0.9\paperheight}}{output_11_0.png}
    \end{center}
    { \hspace*{\fill} \\}
    
    \subsection{TensorFlow Graph}\label{tensorflow-graph}

The entire purpose of TensorFlow is to have a so-called computational
graph that can be executed much more efficiently than if the same
calculations were to be performed directly in Python. TensorFlow can be
more efficient than NumPy because TensorFlow knows the entire
computation graph that must be executed, while NumPy only knows the
computation of a single mathematical operation at a time.

TensorFlow can also automatically calculate the gradients that are
needed to optimize the variables of the graph so as to make the model
perform better. This is because the graph is a combination of simple
mathematical expressions so the gradient of the entire graph can be
calculated using the chain-rule for derivatives.

A TensorFlow graph consists of the following parts which will be
detailed below:

\begin{itemize}
\tightlist
\item
  Placeholder variables used for inputting data to the graph.
\item
  Variables that are going to be optimized so as to make the
  convolutional network perform better.
\item
  The mathematical formulas for the convolutional network.
\item
  A cost measure that can be used to guide the optimization of the
  variables.
\item
  An optimization method which updates the variables.
\end{itemize}

    \subsubsection{Helper-functions for creating new
variables}\label{helper-functions-for-creating-new-variables}

    Functions for creating new TensorFlow variables in the given shape and
initializing them with random values. Note that the initialization is
not actually done at this point, it is merely being defined in the
TensorFlow graph.

    \begin{Verbatim}[commandchars=\\\{\}]
{\color{incolor}In [{\color{incolor}7}]:} \PY{k}{def} \PY{n+nf}{new\PYZus{}weights}\PY{p}{(}\PY{n}{shape}\PY{p}{)}\PY{p}{:}
            \PY{k}{return} \PY{n}{tf}\PY{o}{.}\PY{n}{Variable}\PY{p}{(}\PY{n}{tf}\PY{o}{.}\PY{n}{truncated\PYZus{}normal}\PY{p}{(}\PY{n}{shape}\PY{p}{,} \PY{n}{stddev}\PY{o}{=}\PY{l+m+mf}{0.05}\PY{p}{)}\PY{p}{)}
\end{Verbatim}


    \begin{Verbatim}[commandchars=\\\{\}]
{\color{incolor}In [{\color{incolor}8}]:} \PY{k}{def} \PY{n+nf}{new\PYZus{}biases}\PY{p}{(}\PY{n}{length}\PY{p}{)}\PY{p}{:}
            \PY{k}{return} \PY{n}{tf}\PY{o}{.}\PY{n}{Variable}\PY{p}{(}\PY{n}{tf}\PY{o}{.}\PY{n}{constant}\PY{p}{(}\PY{l+m+mf}{0.05}\PY{p}{,} \PY{n}{shape}\PY{o}{=}\PY{p}{[}\PY{n}{length}\PY{p}{]}\PY{p}{)}\PY{p}{)}
\end{Verbatim}


    \subsubsection{Helper-function for creating a new Convolutional
Layer}\label{helper-function-for-creating-a-new-convolutional-layer}

    This function creates a new convolutional layer in the computational
graph for TensorFlow. Nothing is actually calculated here, we are just
adding the mathematical formulas to the TensorFlow graph.

It is assumed that the input is a 4-dim tensor with the following
dimensions:

\begin{enumerate}
\def\labelenumi{\arabic{enumi}.}
\tightlist
\item
  Image number.
\item
  Y-axis of each image.
\item
  X-axis of each image.
\item
  Channels of each image.
\end{enumerate}

Note that the input channels may either be colour-channels, or it may be
filter-channels if the input is produced from a previous convolutional
layer.

The output is another 4-dim tensor with the following dimensions:

\begin{enumerate}
\def\labelenumi{\arabic{enumi}.}
\tightlist
\item
  Image number, same as input.
\item
  Y-axis of each image. If 2x2 pooling is used, then the height and
  width of the input images is divided by 2.
\item
  X-axis of each image. Ditto.
\item
  Channels produced by the convolutional filters.
\end{enumerate}

    \begin{Verbatim}[commandchars=\\\{\}]
{\color{incolor}In [{\color{incolor}9}]:} \PY{k}{def} \PY{n+nf}{new\PYZus{}conv\PYZus{}layer}\PY{p}{(}\PY{n+nb}{input}\PY{p}{,}              \PY{c+c1}{\PYZsh{} The previous layer.}
                           \PY{n}{num\PYZus{}input\PYZus{}channels}\PY{p}{,} \PY{c+c1}{\PYZsh{} Num. channels in prev. layer.}
                           \PY{n}{filter\PYZus{}size}\PY{p}{,}        \PY{c+c1}{\PYZsh{} Width and height of each filter.}
                           \PY{n}{num\PYZus{}filters}\PY{p}{,}        \PY{c+c1}{\PYZsh{} Number of filters.}
                           \PY{n}{use\PYZus{}pooling}\PY{o}{=}\PY{k+kc}{True}\PY{p}{)}\PY{p}{:}  \PY{c+c1}{\PYZsh{} Use 2x2 max\PYZhy{}pooling.}
        
            \PY{c+c1}{\PYZsh{} Shape of the filter\PYZhy{}weights for the convolution.}
            \PY{c+c1}{\PYZsh{} This format is determined by the TensorFlow API.}
            \PY{n}{shape} \PY{o}{=} \PY{p}{[}\PY{n}{filter\PYZus{}size}\PY{p}{,} \PY{n}{filter\PYZus{}size}\PY{p}{,} \PY{n}{num\PYZus{}input\PYZus{}channels}\PY{p}{,} \PY{n}{num\PYZus{}filters}\PY{p}{]}
        
            \PY{c+c1}{\PYZsh{} Create new weights aka. filters with the given shape.}
            \PY{n}{weights} \PY{o}{=} \PY{n}{new\PYZus{}weights}\PY{p}{(}\PY{n}{shape}\PY{o}{=}\PY{n}{shape}\PY{p}{)}
        
            \PY{c+c1}{\PYZsh{} Create new biases, one for each filter.}
            \PY{n}{biases} \PY{o}{=} \PY{n}{new\PYZus{}biases}\PY{p}{(}\PY{n}{length}\PY{o}{=}\PY{n}{num\PYZus{}filters}\PY{p}{)}
        
            \PY{c+c1}{\PYZsh{} Create the TensorFlow operation for convolution.}
            \PY{c+c1}{\PYZsh{} Note the strides are set to 1 in all dimensions.}
            \PY{c+c1}{\PYZsh{} The first and last stride must always be 1,}
            \PY{c+c1}{\PYZsh{} because the first is for the image\PYZhy{}number and}
            \PY{c+c1}{\PYZsh{} the last is for the input\PYZhy{}channel.}
            \PY{c+c1}{\PYZsh{} But e.g. strides=[1, 2, 2, 1] would mean that the filter}
            \PY{c+c1}{\PYZsh{} is moved 2 pixels across the x\PYZhy{} and y\PYZhy{}axis of the image.}
            \PY{c+c1}{\PYZsh{} The padding is set to \PYZsq{}SAME\PYZsq{} which means the input image}
            \PY{c+c1}{\PYZsh{} is padded with zeroes so the size of the output is the same.}
            \PY{n}{layer} \PY{o}{=} \PY{n}{tf}\PY{o}{.}\PY{n}{nn}\PY{o}{.}\PY{n}{conv2d}\PY{p}{(}\PY{n+nb}{input}\PY{o}{=}\PY{n+nb}{input}\PY{p}{,}
                                 \PY{n+nb}{filter}\PY{o}{=}\PY{n}{weights}\PY{p}{,}
                                 \PY{n}{strides}\PY{o}{=}\PY{p}{[}\PY{l+m+mi}{1}\PY{p}{,} \PY{l+m+mi}{1}\PY{p}{,} \PY{l+m+mi}{1}\PY{p}{,} \PY{l+m+mi}{1}\PY{p}{]}\PY{p}{,}
                                 \PY{n}{padding}\PY{o}{=}\PY{l+s+s1}{\PYZsq{}}\PY{l+s+s1}{SAME}\PY{l+s+s1}{\PYZsq{}}\PY{p}{)}
        
            \PY{c+c1}{\PYZsh{} Add the biases to the results of the convolution.}
            \PY{c+c1}{\PYZsh{} A bias\PYZhy{}value is added to each filter\PYZhy{}channel.}
            \PY{n}{layer} \PY{o}{+}\PY{o}{=} \PY{n}{biases}
        
            \PY{c+c1}{\PYZsh{} Use pooling to down\PYZhy{}sample the image resolution?}
            \PY{k}{if} \PY{n}{use\PYZus{}pooling}\PY{p}{:}
                \PY{c+c1}{\PYZsh{} This is 2x2 max\PYZhy{}pooling, which means that we}
                \PY{c+c1}{\PYZsh{} consider 2x2 windows and select the largest value}
                \PY{c+c1}{\PYZsh{} in each window. Then we move 2 pixels to the next window.}
                \PY{n}{layer} \PY{o}{=} \PY{n}{tf}\PY{o}{.}\PY{n}{nn}\PY{o}{.}\PY{n}{max\PYZus{}pool}\PY{p}{(}\PY{n}{value}\PY{o}{=}\PY{n}{layer}\PY{p}{,}
                                       \PY{n}{ksize}\PY{o}{=}\PY{p}{[}\PY{l+m+mi}{1}\PY{p}{,} \PY{l+m+mi}{2}\PY{p}{,} \PY{l+m+mi}{2}\PY{p}{,} \PY{l+m+mi}{1}\PY{p}{]}\PY{p}{,}
                                       \PY{n}{strides}\PY{o}{=}\PY{p}{[}\PY{l+m+mi}{1}\PY{p}{,} \PY{l+m+mi}{2}\PY{p}{,} \PY{l+m+mi}{2}\PY{p}{,} \PY{l+m+mi}{1}\PY{p}{]}\PY{p}{,}
                                       \PY{n}{padding}\PY{o}{=}\PY{l+s+s1}{\PYZsq{}}\PY{l+s+s1}{SAME}\PY{l+s+s1}{\PYZsq{}}\PY{p}{)}
        
            \PY{c+c1}{\PYZsh{} Rectified Linear Unit (ReLU).}
            \PY{c+c1}{\PYZsh{} It calculates max(x, 0) for each input pixel x.}
            \PY{c+c1}{\PYZsh{} This adds some non\PYZhy{}linearity to the formula and allows us}
            \PY{c+c1}{\PYZsh{} to learn more complicated functions.}
            \PY{n}{layer} \PY{o}{=} \PY{n}{tf}\PY{o}{.}\PY{n}{nn}\PY{o}{.}\PY{n}{relu}\PY{p}{(}\PY{n}{layer}\PY{p}{)}
        
            \PY{c+c1}{\PYZsh{} Note that ReLU is normally executed before the pooling,}
            \PY{c+c1}{\PYZsh{} but since relu(max\PYZus{}pool(x)) == max\PYZus{}pool(relu(x)) we can}
            \PY{c+c1}{\PYZsh{} save 75\PYZpc{} of the relu\PYZhy{}operations by max\PYZhy{}pooling first.}
        
            \PY{c+c1}{\PYZsh{} We return both the resulting layer and the filter\PYZhy{}weights}
            \PY{c+c1}{\PYZsh{} because we will plot the weights later.}
            \PY{k}{return} \PY{n}{layer}\PY{p}{,} \PY{n}{weights}
\end{Verbatim}


    \subsubsection{Helper-function for flattening a
layer}\label{helper-function-for-flattening-a-layer}

A convolutional layer produces an output tensor with 4 dimensions. We
will add fully-connected layers after the convolution layers, so we need
to reduce the 4-dim tensor to 2-dim which can be used as input to the
fully-connected layer.

    \begin{Verbatim}[commandchars=\\\{\}]
{\color{incolor}In [{\color{incolor}10}]:} \PY{k}{def} \PY{n+nf}{flatten\PYZus{}layer}\PY{p}{(}\PY{n}{layer}\PY{p}{)}\PY{p}{:}
             \PY{c+c1}{\PYZsh{} Get the shape of the input layer.}
             \PY{n}{layer\PYZus{}shape} \PY{o}{=} \PY{n}{layer}\PY{o}{.}\PY{n}{get\PYZus{}shape}\PY{p}{(}\PY{p}{)}
         
             \PY{c+c1}{\PYZsh{} The shape of the input layer is assumed to be:}
             \PY{c+c1}{\PYZsh{} layer\PYZus{}shape == [num\PYZus{}images, img\PYZus{}height, img\PYZus{}width, num\PYZus{}channels]}
         
             \PY{c+c1}{\PYZsh{} The number of features is: img\PYZus{}height * img\PYZus{}width * num\PYZus{}channels}
             \PY{c+c1}{\PYZsh{} We can use a function from TensorFlow to calculate this.}
             \PY{n}{num\PYZus{}features} \PY{o}{=} \PY{n}{layer\PYZus{}shape}\PY{p}{[}\PY{l+m+mi}{1}\PY{p}{:}\PY{l+m+mi}{4}\PY{p}{]}\PY{o}{.}\PY{n}{num\PYZus{}elements}\PY{p}{(}\PY{p}{)}
             
             \PY{c+c1}{\PYZsh{} Reshape the layer to [num\PYZus{}images, num\PYZus{}features].}
             \PY{c+c1}{\PYZsh{} Note that we just set the size of the second dimension}
             \PY{c+c1}{\PYZsh{} to num\PYZus{}features and the size of the first dimension to \PYZhy{}1}
             \PY{c+c1}{\PYZsh{} which means the size in that dimension is calculated}
             \PY{c+c1}{\PYZsh{} so the total size of the tensor is unchanged from the reshaping.}
             \PY{n}{layer\PYZus{}flat} \PY{o}{=} \PY{n}{tf}\PY{o}{.}\PY{n}{reshape}\PY{p}{(}\PY{n}{layer}\PY{p}{,} \PY{p}{[}\PY{o}{\PYZhy{}}\PY{l+m+mi}{1}\PY{p}{,} \PY{n}{num\PYZus{}features}\PY{p}{]}\PY{p}{)}
         
             \PY{c+c1}{\PYZsh{} The shape of the flattened layer is now:}
             \PY{c+c1}{\PYZsh{} [num\PYZus{}images, img\PYZus{}height * img\PYZus{}width * num\PYZus{}channels]}
         
             \PY{c+c1}{\PYZsh{} Return both the flattened layer and the number of features.}
             \PY{k}{return} \PY{n}{layer\PYZus{}flat}\PY{p}{,} \PY{n}{num\PYZus{}features}
\end{Verbatim}


    \subsubsection{Helper-function for creating a new Fully-Connected
Layer}\label{helper-function-for-creating-a-new-fully-connected-layer}

    This function creates a new fully-connected layer in the computational
graph for TensorFlow. Nothing is actually calculated here, we are just
adding the mathematical formulas to the TensorFlow graph.

It is assumed that the input is a 2-dim tensor of shape
\texttt{{[}num\_images,\ num\_inputs{]}}. The output is a 2-dim tensor
of shape \texttt{{[}num\_images,\ num\_outputs{]}}.

    \begin{Verbatim}[commandchars=\\\{\}]
{\color{incolor}In [{\color{incolor}11}]:} \PY{k}{def} \PY{n+nf}{new\PYZus{}fc\PYZus{}layer}\PY{p}{(}\PY{n+nb}{input}\PY{p}{,}          \PY{c+c1}{\PYZsh{} The previous layer.}
                          \PY{n}{num\PYZus{}inputs}\PY{p}{,}     \PY{c+c1}{\PYZsh{} Num. inputs from prev. layer.}
                          \PY{n}{num\PYZus{}outputs}\PY{p}{,}    \PY{c+c1}{\PYZsh{} Num. outputs.}
                          \PY{n}{use\PYZus{}relu}\PY{o}{=}\PY{k+kc}{True}\PY{p}{)}\PY{p}{:} \PY{c+c1}{\PYZsh{} Use Rectified Linear Unit (ReLU)?}
         
             \PY{c+c1}{\PYZsh{} Create new weights and biases.}
             \PY{n}{weights} \PY{o}{=} \PY{n}{new\PYZus{}weights}\PY{p}{(}\PY{n}{shape}\PY{o}{=}\PY{p}{[}\PY{n}{num\PYZus{}inputs}\PY{p}{,} \PY{n}{num\PYZus{}outputs}\PY{p}{]}\PY{p}{)}
             \PY{n}{biases} \PY{o}{=} \PY{n}{new\PYZus{}biases}\PY{p}{(}\PY{n}{length}\PY{o}{=}\PY{n}{num\PYZus{}outputs}\PY{p}{)}
         
             \PY{c+c1}{\PYZsh{} Calculate the layer as the matrix multiplication of}
             \PY{c+c1}{\PYZsh{} the input and weights, and then add the bias\PYZhy{}values.}
             \PY{n}{layer} \PY{o}{=} \PY{n}{tf}\PY{o}{.}\PY{n}{matmul}\PY{p}{(}\PY{n+nb}{input}\PY{p}{,} \PY{n}{weights}\PY{p}{)} \PY{o}{+} \PY{n}{biases}
         
             \PY{c+c1}{\PYZsh{} Use ReLU?}
             \PY{k}{if} \PY{n}{use\PYZus{}relu}\PY{p}{:}
                 \PY{n}{layer} \PY{o}{=} \PY{n}{tf}\PY{o}{.}\PY{n}{nn}\PY{o}{.}\PY{n}{relu}\PY{p}{(}\PY{n}{layer}\PY{p}{)}
         
             \PY{k}{return} \PY{n}{layer}
\end{Verbatim}


    \subsubsection{Placeholder variables}\label{placeholder-variables}

    Placeholder variables serve as the input to the TensorFlow computational
graph that we may change each time we execute the graph. We call this
feeding the placeholder variables and it is demonstrated further below.

First we define the placeholder variable for the input images. This
allows us to change the images that are input to the TensorFlow graph.
This is a so-called tensor, which just means that it is a
multi-dimensional vector or matrix. The data-type is set to
\texttt{float32} and the shape is set to
\texttt{{[}None,\ img\_size\_flat{]}}, where \texttt{None} means that
the tensor may hold an arbitrary number of images with each image being
a vector of length \texttt{img\_size\_flat}.

    \begin{Verbatim}[commandchars=\\\{\}]
{\color{incolor}In [{\color{incolor}12}]:} \PY{n}{x} \PY{o}{=} \PY{n}{tf}\PY{o}{.}\PY{n}{placeholder}\PY{p}{(}\PY{n}{tf}\PY{o}{.}\PY{n}{float32}\PY{p}{,} \PY{n}{shape}\PY{o}{=}\PY{p}{[}\PY{k+kc}{None}\PY{p}{,} \PY{n}{img\PYZus{}size\PYZus{}flat}\PY{p}{]}\PY{p}{,} \PY{n}{name}\PY{o}{=}\PY{l+s+s1}{\PYZsq{}}\PY{l+s+s1}{x}\PY{l+s+s1}{\PYZsq{}}\PY{p}{)}
\end{Verbatim}


    The convolutional layers expect \texttt{x} to be encoded as a 4-dim
tensor so we have to reshape it so its shape is instead
\texttt{{[}num\_images,\ img\_height,\ img\_width,\ num\_channels{]}}.
Note that \texttt{img\_height\ ==\ img\_width\ ==\ img\_size} and
\texttt{num\_images} can be inferred automatically by using -1 for the
size of the first dimension. So the reshape operation is:

    \begin{Verbatim}[commandchars=\\\{\}]
{\color{incolor}In [{\color{incolor}13}]:} \PY{n}{x\PYZus{}image} \PY{o}{=} \PY{n}{tf}\PY{o}{.}\PY{n}{reshape}\PY{p}{(}\PY{n}{x}\PY{p}{,} \PY{p}{[}\PY{o}{\PYZhy{}}\PY{l+m+mi}{1}\PY{p}{,} \PY{n}{img\PYZus{}size}\PY{p}{,} \PY{n}{img\PYZus{}size}\PY{p}{,} \PY{n}{num\PYZus{}channels}\PY{p}{]}\PY{p}{)}
\end{Verbatim}


    Next we have the placeholder variable for the true labels associated
with the images that were input in the placeholder variable \texttt{x}.
The shape of this placeholder variable is
\texttt{{[}None,\ num\_classes{]}} which means it may hold an arbitrary
number of labels and each label is a vector of length
\texttt{num\_classes}.

    \begin{Verbatim}[commandchars=\\\{\}]
{\color{incolor}In [{\color{incolor}14}]:} \PY{n}{y\PYZus{}true} \PY{o}{=} \PY{n}{tf}\PY{o}{.}\PY{n}{placeholder}\PY{p}{(}\PY{n}{tf}\PY{o}{.}\PY{n}{float32}\PY{p}{,} \PY{n}{shape}\PY{o}{=}\PY{p}{[}\PY{k+kc}{None}\PY{p}{,} \PY{n}{num\PYZus{}classes}\PY{p}{]}\PY{p}{,} \PY{n}{name}\PY{o}{=}\PY{l+s+s1}{\PYZsq{}}\PY{l+s+s1}{y\PYZus{}true}\PY{l+s+s1}{\PYZsq{}}\PY{p}{)}
\end{Verbatim}


    We could also have a placeholder variable for the class-number, but we
will instead calculate it using argmax. Note that this is a TensorFlow
operator so nothing is calculated at this point.

    \begin{Verbatim}[commandchars=\\\{\}]
{\color{incolor}In [{\color{incolor}15}]:} \PY{n}{y\PYZus{}true\PYZus{}cls} \PY{o}{=} \PY{n}{tf}\PY{o}{.}\PY{n}{argmax}\PY{p}{(}\PY{n}{y\PYZus{}true}\PY{p}{,} \PY{n}{dimension}\PY{o}{=}\PY{l+m+mi}{1}\PY{p}{)}
\end{Verbatim}


    \begin{Verbatim}[commandchars=\\\{\}]
WARNING:tensorflow:From <ipython-input-15-71ccadb4572d>:1: calling argmax (from tensorflow.python.ops.math\_ops) with dimension is deprecated and will be removed in a future version.
Instructions for updating:
Use the `axis` argument instead

    \end{Verbatim}

    \subsubsection{Convolutional Layer 1}\label{convolutional-layer-1}

Create the first convolutional layer. It takes \texttt{x\_image} as
input and creates \texttt{num\_filters1} different filters, each having
width and height equal to \texttt{filter\_size1}. Finally we wish to
down-sample the image so it is half the size by using 2x2 max-pooling.

    \begin{Verbatim}[commandchars=\\\{\}]
{\color{incolor}In [{\color{incolor}16}]:} \PY{n}{layer\PYZus{}conv1}\PY{p}{,} \PY{n}{weights\PYZus{}conv1} \PY{o}{=} \PYZbs{}
             \PY{n}{new\PYZus{}conv\PYZus{}layer}\PY{p}{(}\PY{n+nb}{input}\PY{o}{=}\PY{n}{x\PYZus{}image}\PY{p}{,}
                            \PY{n}{num\PYZus{}input\PYZus{}channels}\PY{o}{=}\PY{n}{num\PYZus{}channels}\PY{p}{,}
                            \PY{n}{filter\PYZus{}size}\PY{o}{=}\PY{n}{filter\PYZus{}size1}\PY{p}{,}
                            \PY{n}{num\PYZus{}filters}\PY{o}{=}\PY{n}{num\PYZus{}filters1}\PY{p}{,}
                            \PY{n}{use\PYZus{}pooling}\PY{o}{=}\PY{k+kc}{True}\PY{p}{)}
\end{Verbatim}


    \begin{Verbatim}[commandchars=\\\{\}]
{\color{incolor}In [{\color{incolor}17}]:} \PY{n}{layer\PYZus{}conv1}
\end{Verbatim}


\begin{Verbatim}[commandchars=\\\{\}]
{\color{outcolor}Out[{\color{outcolor}17}]:} <tf.Tensor 'Relu:0' shape=(?, 64, 64, 32) dtype=float32>
\end{Verbatim}
            
    \subsubsection{Convolutional Layers 2 and
3}\label{convolutional-layers-2-and-3}

Create the second and third convolutional layers, which take as input
the output from the first and second convolutional layer respectively.
The number of input channels corresponds to the number of filters in the
previous convolutional layer.

    \begin{Verbatim}[commandchars=\\\{\}]
{\color{incolor}In [{\color{incolor}18}]:} \PY{n}{layer\PYZus{}conv2}\PY{p}{,} \PY{n}{weights\PYZus{}conv2} \PY{o}{=} \PYZbs{}
             \PY{n}{new\PYZus{}conv\PYZus{}layer}\PY{p}{(}\PY{n+nb}{input}\PY{o}{=}\PY{n}{layer\PYZus{}conv1}\PY{p}{,}
                            \PY{n}{num\PYZus{}input\PYZus{}channels}\PY{o}{=}\PY{n}{num\PYZus{}filters1}\PY{p}{,}
                            \PY{n}{filter\PYZus{}size}\PY{o}{=}\PY{n}{filter\PYZus{}size2}\PY{p}{,}
                            \PY{n}{num\PYZus{}filters}\PY{o}{=}\PY{n}{num\PYZus{}filters2}\PY{p}{,}
                            \PY{n}{use\PYZus{}pooling}\PY{o}{=}\PY{k+kc}{True}\PY{p}{)}
\end{Verbatim}


    \begin{Verbatim}[commandchars=\\\{\}]
{\color{incolor}In [{\color{incolor}19}]:} \PY{n}{layer\PYZus{}conv2}
\end{Verbatim}


\begin{Verbatim}[commandchars=\\\{\}]
{\color{outcolor}Out[{\color{outcolor}19}]:} <tf.Tensor 'Relu\_1:0' shape=(?, 32, 32, 32) dtype=float32>
\end{Verbatim}
            
    \begin{Verbatim}[commandchars=\\\{\}]
{\color{incolor}In [{\color{incolor}20}]:} \PY{n}{layer\PYZus{}conv3}\PY{p}{,} \PY{n}{weights\PYZus{}conv3} \PY{o}{=} \PYZbs{}
             \PY{n}{new\PYZus{}conv\PYZus{}layer}\PY{p}{(}\PY{n+nb}{input}\PY{o}{=}\PY{n}{layer\PYZus{}conv2}\PY{p}{,}
                            \PY{n}{num\PYZus{}input\PYZus{}channels}\PY{o}{=}\PY{n}{num\PYZus{}filters2}\PY{p}{,}
                            \PY{n}{filter\PYZus{}size}\PY{o}{=}\PY{n}{filter\PYZus{}size3}\PY{p}{,}
                            \PY{n}{num\PYZus{}filters}\PY{o}{=}\PY{n}{num\PYZus{}filters3}\PY{p}{,}
                            \PY{n}{use\PYZus{}pooling}\PY{o}{=}\PY{k+kc}{True}\PY{p}{)}
\end{Verbatim}


    \begin{Verbatim}[commandchars=\\\{\}]
{\color{incolor}In [{\color{incolor}21}]:} \PY{n}{layer\PYZus{}conv3}
\end{Verbatim}


\begin{Verbatim}[commandchars=\\\{\}]
{\color{outcolor}Out[{\color{outcolor}21}]:} <tf.Tensor 'Relu\_2:0' shape=(?, 16, 16, 64) dtype=float32>
\end{Verbatim}
            
    \subsubsection{Flatten Layer}\label{flatten-layer}

The convolutional layers output 4-dim tensors. We now wish to use these
as input in a fully-connected network, which requires for the tensors to
be reshaped or flattened to 2-dim tensors.

    \begin{Verbatim}[commandchars=\\\{\}]
{\color{incolor}In [{\color{incolor}22}]:} \PY{n}{layer\PYZus{}flat}\PY{p}{,} \PY{n}{num\PYZus{}features} \PY{o}{=} \PY{n}{flatten\PYZus{}layer}\PY{p}{(}\PY{n}{layer\PYZus{}conv3}\PY{p}{)}
\end{Verbatim}


    \begin{Verbatim}[commandchars=\\\{\}]
{\color{incolor}In [{\color{incolor}23}]:} \PY{n}{layer\PYZus{}flat}
\end{Verbatim}


\begin{Verbatim}[commandchars=\\\{\}]
{\color{outcolor}Out[{\color{outcolor}23}]:} <tf.Tensor 'Reshape\_1:0' shape=(?, 16384) dtype=float32>
\end{Verbatim}
            
    \begin{Verbatim}[commandchars=\\\{\}]
{\color{incolor}In [{\color{incolor}24}]:} \PY{n}{num\PYZus{}features}
\end{Verbatim}


\begin{Verbatim}[commandchars=\\\{\}]
{\color{outcolor}Out[{\color{outcolor}24}]:} 16384
\end{Verbatim}
            
    \subsubsection{Fully-Connected Layer 1}\label{fully-connected-layer-1}

Add a fully-connected layer to the network. The input is the flattened
layer from the previous convolution. The number of neurons or nodes in
the fully-connected layer is \texttt{fc\_size}. ReLU is used so we can
learn non-linear relations.

    \begin{Verbatim}[commandchars=\\\{\}]
{\color{incolor}In [{\color{incolor}25}]:} \PY{n}{layer\PYZus{}fc1} \PY{o}{=} \PY{n}{new\PYZus{}fc\PYZus{}layer}\PY{p}{(}\PY{n+nb}{input}\PY{o}{=}\PY{n}{layer\PYZus{}flat}\PY{p}{,}
                                  \PY{n}{num\PYZus{}inputs}\PY{o}{=}\PY{n}{num\PYZus{}features}\PY{p}{,}
                                  \PY{n}{num\PYZus{}outputs}\PY{o}{=}\PY{n}{fc\PYZus{}size}\PY{p}{,}
                                  \PY{n}{use\PYZus{}relu}\PY{o}{=}\PY{k+kc}{True}\PY{p}{)}
\end{Verbatim}


    Check that the output of the fully-connected layer is a tensor with
shape (?, 128) where the ? means there is an arbitrary number of images
and \texttt{fc\_size} == 128.

    \begin{Verbatim}[commandchars=\\\{\}]
{\color{incolor}In [{\color{incolor}26}]:} \PY{n}{layer\PYZus{}fc1}
\end{Verbatim}


\begin{Verbatim}[commandchars=\\\{\}]
{\color{outcolor}Out[{\color{outcolor}26}]:} <tf.Tensor 'Relu\_3:0' shape=(?, 128) dtype=float32>
\end{Verbatim}
            
    \subsubsection{Fully-Connected Layer 2}\label{fully-connected-layer-2}

Add another fully-connected layer that outputs vectors of length
num\_classes for determining which of the classes the input image
belongs to. Note that ReLU is not used in this layer.

    \begin{Verbatim}[commandchars=\\\{\}]
{\color{incolor}In [{\color{incolor}27}]:} \PY{n}{layer\PYZus{}fc2} \PY{o}{=} \PY{n}{new\PYZus{}fc\PYZus{}layer}\PY{p}{(}\PY{n+nb}{input}\PY{o}{=}\PY{n}{layer\PYZus{}fc1}\PY{p}{,}
                                  \PY{n}{num\PYZus{}inputs}\PY{o}{=}\PY{n}{fc\PYZus{}size}\PY{p}{,}
                                  \PY{n}{num\PYZus{}outputs}\PY{o}{=}\PY{n}{num\PYZus{}classes}\PY{p}{,}
                                  \PY{n}{use\PYZus{}relu}\PY{o}{=}\PY{k+kc}{False}\PY{p}{)}
\end{Verbatim}


    \begin{Verbatim}[commandchars=\\\{\}]
{\color{incolor}In [{\color{incolor}28}]:} \PY{n}{layer\PYZus{}fc2}
\end{Verbatim}


\begin{Verbatim}[commandchars=\\\{\}]
{\color{outcolor}Out[{\color{outcolor}28}]:} <tf.Tensor 'add\_4:0' shape=(?, 2) dtype=float32>
\end{Verbatim}
            
    \subsubsection{Predicted Class}\label{predicted-class}

    The second fully-connected layer estimates how likely it is that the
input image belongs to each of the 2 classes. However, these estimates
are a bit rough and difficult to interpret because the numbers may be
very small or large, so we want to normalize them so that each element
is limited between zero and one and the all the elements sum to one.
This is calculated using the so-called softmax function and the result
is stored in \texttt{y\_pred}.

    \begin{Verbatim}[commandchars=\\\{\}]
{\color{incolor}In [{\color{incolor}29}]:} \PY{n}{y\PYZus{}pred} \PY{o}{=} \PY{n}{tf}\PY{o}{.}\PY{n}{nn}\PY{o}{.}\PY{n}{softmax}\PY{p}{(}\PY{n}{layer\PYZus{}fc2}\PY{p}{)}
\end{Verbatim}


    The class-number is the index of the largest element.

    \begin{Verbatim}[commandchars=\\\{\}]
{\color{incolor}In [{\color{incolor}30}]:} \PY{n}{y\PYZus{}pred\PYZus{}cls} \PY{o}{=} \PY{n}{tf}\PY{o}{.}\PY{n}{argmax}\PY{p}{(}\PY{n}{y\PYZus{}pred}\PY{p}{,} \PY{n}{dimension}\PY{o}{=}\PY{l+m+mi}{1}\PY{p}{)}
\end{Verbatim}


    \subsubsection{Cost-function to be
optimized}\label{cost-function-to-be-optimized}

    To make the model better at classifying the input images, we must
somehow change the variables for all the network layers. To do this we
first need to know how well the model currently performs by comparing
the predicted output of the model \texttt{y\_pred} to the desired output
\texttt{y\_true}.

The cross-entropy is a performance measure used in classification. The
cross-entropy is a continuous function that is always positive and if
the predicted output of the model exactly matches the desired output
then the cross-entropy equals zero. The goal of optimization is
therefore to minimize the cross-entropy so it gets as close to zero as
possible by changing the variables of the network layers.

TensorFlow has a built-in function for calculating the cross-entropy.
Note that the function calculates the softmax internally so we must use
the output of \texttt{layer\_fc2} directly rather than \texttt{y\_pred}
which has already had the softmax applied.

    \begin{Verbatim}[commandchars=\\\{\}]
{\color{incolor}In [{\color{incolor}31}]:} \PY{n}{cross\PYZus{}entropy} \PY{o}{=} \PY{n}{tf}\PY{o}{.}\PY{n}{nn}\PY{o}{.}\PY{n}{softmax\PYZus{}cross\PYZus{}entropy\PYZus{}with\PYZus{}logits}\PY{p}{(}\PY{n}{logits}\PY{o}{=}\PY{n}{layer\PYZus{}fc2}\PY{p}{,}
                                                                 \PY{n}{labels}\PY{o}{=}\PY{n}{y\PYZus{}true}\PY{p}{)}
\end{Verbatim}


    \begin{Verbatim}[commandchars=\\\{\}]
WARNING:tensorflow:From <ipython-input-31-2dd067a7547b>:2: softmax\_cross\_entropy\_with\_logits (from tensorflow.python.ops.nn\_ops) is deprecated and will be removed in a future version.
Instructions for updating:

Future major versions of TensorFlow will allow gradients to flow
into the labels input on backprop by default.

See @\{tf.nn.softmax\_cross\_entropy\_with\_logits\_v2\}.


    \end{Verbatim}

    We have now calculated the cross-entropy for each of the image
classifications so we have a measure of how well the model performs on
each image individually. But in order to use the cross-entropy to guide
the optimization of the model's variables we need a single scalar value,
so we simply take the average of the cross-entropy for all the image
classifications.

    \begin{Verbatim}[commandchars=\\\{\}]
{\color{incolor}In [{\color{incolor}32}]:} \PY{n}{cost} \PY{o}{=} \PY{n}{tf}\PY{o}{.}\PY{n}{reduce\PYZus{}mean}\PY{p}{(}\PY{n}{cross\PYZus{}entropy}\PY{p}{)}
\end{Verbatim}


    \subsubsection{Optimization Method}\label{optimization-method}

    Now that we have a cost measure that must be minimized, we can then
create an optimizer. In this case it is the \texttt{AdamOptimizer} which
is an advanced form of Gradient Descent.

Note that optimization is not performed at this point. In fact, nothing
is calculated at all, we just add the optimizer-object to the TensorFlow
graph for later execution.

    \begin{Verbatim}[commandchars=\\\{\}]
{\color{incolor}In [{\color{incolor}33}]:} \PY{n}{optimizer} \PY{o}{=} \PY{n}{tf}\PY{o}{.}\PY{n}{train}\PY{o}{.}\PY{n}{AdamOptimizer}\PY{p}{(}\PY{n}{learning\PYZus{}rate}\PY{o}{=}\PY{l+m+mf}{1e\PYZhy{}4}\PY{p}{)}\PY{o}{.}\PY{n}{minimize}\PY{p}{(}\PY{n}{cost}\PY{p}{)}
\end{Verbatim}


    \subsubsection{Performance Measures}\label{performance-measures}

    We need a few more performance measures to display the progress to the
user.

This is a vector of booleans whether the predicted class equals the true
class of each image.

    \begin{Verbatim}[commandchars=\\\{\}]
{\color{incolor}In [{\color{incolor}34}]:} \PY{n}{correct\PYZus{}prediction} \PY{o}{=} \PY{n}{tf}\PY{o}{.}\PY{n}{equal}\PY{p}{(}\PY{n}{y\PYZus{}pred\PYZus{}cls}\PY{p}{,} \PY{n}{y\PYZus{}true\PYZus{}cls}\PY{p}{)}
\end{Verbatim}


    This calculates the classification accuracy by first type-casting the
vector of booleans to floats, so that False becomes 0 and True becomes
1, and then calculating the average of these numbers.

    \begin{Verbatim}[commandchars=\\\{\}]
{\color{incolor}In [{\color{incolor}35}]:} \PY{n}{accuracy} \PY{o}{=} \PY{n}{tf}\PY{o}{.}\PY{n}{reduce\PYZus{}mean}\PY{p}{(}\PY{n}{tf}\PY{o}{.}\PY{n}{cast}\PY{p}{(}\PY{n}{correct\PYZus{}prediction}\PY{p}{,} \PY{n}{tf}\PY{o}{.}\PY{n}{float32}\PY{p}{)}\PY{p}{)}
\end{Verbatim}


    \subsection{TensorFlow Run}\label{tensorflow-run}

    \subsubsection{Create TensorFlow
session}\label{create-tensorflow-session}

Once the TensorFlow graph has been created, we have to create a
TensorFlow session which is used to execute the graph.

    \begin{Verbatim}[commandchars=\\\{\}]
{\color{incolor}In [{\color{incolor}36}]:} \PY{n}{session} \PY{o}{=} \PY{n}{tf}\PY{o}{.}\PY{n}{Session}\PY{p}{(}\PY{p}{)}
\end{Verbatim}


    \subsubsection{Initialize variables}\label{initialize-variables}

The variables for \texttt{weights} and \texttt{biases} must be
initialized before we start optimizing them.

    \begin{Verbatim}[commandchars=\\\{\}]
{\color{incolor}In [{\color{incolor}37}]:} \PY{n}{session}\PY{o}{.}\PY{n}{run}\PY{p}{(}\PY{n}{tf}\PY{o}{.}\PY{n}{initialize\PYZus{}all\PYZus{}variables}\PY{p}{(}\PY{p}{)}\PY{p}{)}
\end{Verbatim}


    \begin{Verbatim}[commandchars=\\\{\}]
WARNING:tensorflow:From C:\textbackslash{}Users\textbackslash{}ameypawar\textbackslash{}Anaconda2\textbackslash{}envs\textbackslash{}python3\textbackslash{}lib\textbackslash{}site-packages\textbackslash{}tensorflow\textbackslash{}python\textbackslash{}util\textbackslash{}tf\_should\_use.py:118: initialize\_all\_variables (from tensorflow.python.ops.variables) is deprecated and will be removed after 2017-03-02.
Instructions for updating:
Use `tf.global\_variables\_initializer` instead.

    \end{Verbatim}

    \subsubsection{Helper-function to perform optimization
iterations}\label{helper-function-to-perform-optimization-iterations}

    It takes a long time to calculate the gradient of the model using the
entirety of a large dataset . We therefore only use a small batch of
images in each iteration of the optimizer.

If your computer crashes or becomes very slow because you run out of
RAM, then you may try and lower this number, but you may then need to
perform more optimization iterations.

    \begin{Verbatim}[commandchars=\\\{\}]
{\color{incolor}In [{\color{incolor}38}]:} \PY{n}{train\PYZus{}batch\PYZus{}size} \PY{o}{=} \PY{n}{batch\PYZus{}size}
\end{Verbatim}


    \begin{Verbatim}[commandchars=\\\{\}]
{\color{incolor}In [{\color{incolor}39}]:} \PY{k}{def} \PY{n+nf}{print\PYZus{}progress}\PY{p}{(}\PY{n}{epoch}\PY{p}{,} \PY{n}{feed\PYZus{}dict\PYZus{}train}\PY{p}{,} \PY{n}{feed\PYZus{}dict\PYZus{}validate}\PY{p}{,} \PY{n}{val\PYZus{}loss}\PY{p}{)}\PY{p}{:}
             \PY{c+c1}{\PYZsh{} Calculate the accuracy on the training\PYZhy{}set.}
             \PY{n}{acc} \PY{o}{=} \PY{n}{session}\PY{o}{.}\PY{n}{run}\PY{p}{(}\PY{n}{accuracy}\PY{p}{,} \PY{n}{feed\PYZus{}dict}\PY{o}{=}\PY{n}{feed\PYZus{}dict\PYZus{}train}\PY{p}{)}
             \PY{n}{val\PYZus{}acc} \PY{o}{=} \PY{n}{session}\PY{o}{.}\PY{n}{run}\PY{p}{(}\PY{n}{accuracy}\PY{p}{,} \PY{n}{feed\PYZus{}dict}\PY{o}{=}\PY{n}{feed\PYZus{}dict\PYZus{}validate}\PY{p}{)}
             \PY{n}{msg} \PY{o}{=} \PY{l+s+s2}{\PYZdq{}}\PY{l+s+s2}{Epoch }\PY{l+s+si}{\PYZob{}0\PYZcb{}}\PY{l+s+s2}{ \PYZhy{}\PYZhy{}\PYZhy{} Training Accuracy: }\PY{l+s+si}{\PYZob{}1:\PYZgt{}6.1\PYZpc{}\PYZcb{}}\PY{l+s+s2}{, Validation Accuracy: }\PY{l+s+si}{\PYZob{}2:\PYZgt{}6.1\PYZpc{}\PYZcb{}}\PY{l+s+s2}{, Validation Loss: }\PY{l+s+si}{\PYZob{}3:.3f\PYZcb{}}\PY{l+s+s2}{\PYZdq{}}
             \PY{n+nb}{print}\PY{p}{(}\PY{n}{msg}\PY{o}{.}\PY{n}{format}\PY{p}{(}\PY{n}{epoch} \PY{o}{+} \PY{l+m+mi}{1}\PY{p}{,} \PY{n}{acc}\PY{p}{,} \PY{n}{val\PYZus{}acc}\PY{p}{,} \PY{n}{val\PYZus{}loss}\PY{p}{)}\PY{p}{)}
\end{Verbatim}


    Function for performing a number of optimization iterations so as to
gradually improve the variables of the network layers. In each
iteration, a new batch of data is selected from the training-set and
then TensorFlow executes the optimizer using those training samples. The
progress is printed every 100 iterations.

    \begin{Verbatim}[commandchars=\\\{\}]
{\color{incolor}In [{\color{incolor}40}]:} \PY{c+c1}{\PYZsh{} Counter for total number of iterations performed so far.}
         \PY{n}{total\PYZus{}iterations} \PY{o}{=} \PY{l+m+mi}{0}
         
         \PY{k}{def} \PY{n+nf}{optimize}\PY{p}{(}\PY{n}{num\PYZus{}iterations}\PY{p}{)}\PY{p}{:}
             \PY{c+c1}{\PYZsh{} Ensure we update the global variable rather than a local copy.}
             \PY{k}{global} \PY{n}{total\PYZus{}iterations}
         
             \PY{c+c1}{\PYZsh{} Start\PYZhy{}time used for printing time\PYZhy{}usage below.}
             \PY{n}{start\PYZus{}time} \PY{o}{=} \PY{n}{time}\PY{o}{.}\PY{n}{time}\PY{p}{(}\PY{p}{)}
             
             \PY{n}{best\PYZus{}val\PYZus{}loss} \PY{o}{=} \PY{n+nb}{float}\PY{p}{(}\PY{l+s+s2}{\PYZdq{}}\PY{l+s+s2}{inf}\PY{l+s+s2}{\PYZdq{}}\PY{p}{)}
             \PY{n}{patience} \PY{o}{=} \PY{l+m+mi}{0}
         
             \PY{k}{for} \PY{n}{i} \PY{o+ow}{in} \PY{n+nb}{range}\PY{p}{(}\PY{n}{total\PYZus{}iterations}\PY{p}{,}
                            \PY{n}{total\PYZus{}iterations} \PY{o}{+} \PY{n}{num\PYZus{}iterations}\PY{p}{)}\PY{p}{:}
         
                 \PY{c+c1}{\PYZsh{} Get a batch of training examples.}
                 \PY{c+c1}{\PYZsh{} x\PYZus{}batch now holds a batch of images and}
                 \PY{c+c1}{\PYZsh{} y\PYZus{}true\PYZus{}batch are the true labels for those images.}
                 \PY{n}{x\PYZus{}batch}\PY{p}{,} \PY{n}{y\PYZus{}true\PYZus{}batch}\PY{p}{,} \PY{n}{\PYZus{}}\PY{p}{,} \PY{n}{cls\PYZus{}batch} \PY{o}{=} \PY{n}{data}\PY{o}{.}\PY{n}{train}\PY{o}{.}\PY{n}{next\PYZus{}batch}\PY{p}{(}\PY{n}{train\PYZus{}batch\PYZus{}size}\PY{p}{)}
                 \PY{n}{x\PYZus{}valid\PYZus{}batch}\PY{p}{,} \PY{n}{y\PYZus{}valid\PYZus{}batch}\PY{p}{,} \PY{n}{\PYZus{}}\PY{p}{,} \PY{n}{valid\PYZus{}cls\PYZus{}batch} \PY{o}{=} \PY{n}{data}\PY{o}{.}\PY{n}{valid}\PY{o}{.}\PY{n}{next\PYZus{}batch}\PY{p}{(}\PY{n}{train\PYZus{}batch\PYZus{}size}\PY{p}{)}
         
                 \PY{c+c1}{\PYZsh{} Convert shape from [num examples, rows, columns, depth]}
                 \PY{c+c1}{\PYZsh{} to [num examples, flattened image shape]}
         
                 \PY{n}{x\PYZus{}batch} \PY{o}{=} \PY{n}{x\PYZus{}batch}\PY{o}{.}\PY{n}{reshape}\PY{p}{(}\PY{n}{train\PYZus{}batch\PYZus{}size}\PY{p}{,} \PY{n}{img\PYZus{}size\PYZus{}flat}\PY{p}{)}
                 \PY{n}{x\PYZus{}valid\PYZus{}batch} \PY{o}{=} \PY{n}{x\PYZus{}valid\PYZus{}batch}\PY{o}{.}\PY{n}{reshape}\PY{p}{(}\PY{n}{train\PYZus{}batch\PYZus{}size}\PY{p}{,} \PY{n}{img\PYZus{}size\PYZus{}flat}\PY{p}{)}
         
                 \PY{c+c1}{\PYZsh{} Put the batch into a dict with the proper names}
                 \PY{c+c1}{\PYZsh{} for placeholder variables in the TensorFlow graph.}
                 \PY{n}{feed\PYZus{}dict\PYZus{}train} \PY{o}{=} \PY{p}{\PYZob{}}\PY{n}{x}\PY{p}{:} \PY{n}{x\PYZus{}batch}\PY{p}{,}
                                    \PY{n}{y\PYZus{}true}\PY{p}{:} \PY{n}{y\PYZus{}true\PYZus{}batch}\PY{p}{\PYZcb{}}
                 
                 \PY{n}{feed\PYZus{}dict\PYZus{}validate} \PY{o}{=} \PY{p}{\PYZob{}}\PY{n}{x}\PY{p}{:} \PY{n}{x\PYZus{}valid\PYZus{}batch}\PY{p}{,}
                                       \PY{n}{y\PYZus{}true}\PY{p}{:} \PY{n}{y\PYZus{}valid\PYZus{}batch}\PY{p}{\PYZcb{}}
         
                 \PY{c+c1}{\PYZsh{} Run the optimizer using this batch of training data.}
                 \PY{c+c1}{\PYZsh{} TensorFlow assigns the variables in feed\PYZus{}dict\PYZus{}train}
                 \PY{c+c1}{\PYZsh{} to the placeholder variables and then runs the optimizer.}
                 \PY{n}{session}\PY{o}{.}\PY{n}{run}\PY{p}{(}\PY{n}{optimizer}\PY{p}{,} \PY{n}{feed\PYZus{}dict}\PY{o}{=}\PY{n}{feed\PYZus{}dict\PYZus{}train}\PY{p}{)}
                 
         
                 \PY{c+c1}{\PYZsh{} Print status at end of each epoch (defined as full pass through training dataset).}
                 \PY{k}{if} \PY{n}{i} \PY{o}{\PYZpc{}} \PY{n+nb}{int}\PY{p}{(}\PY{n}{data}\PY{o}{.}\PY{n}{train}\PY{o}{.}\PY{n}{num\PYZus{}examples}\PY{o}{/}\PY{n}{batch\PYZus{}size}\PY{p}{)} \PY{o}{==} \PY{l+m+mi}{0}\PY{p}{:} 
                     \PY{n}{val\PYZus{}loss} \PY{o}{=} \PY{n}{session}\PY{o}{.}\PY{n}{run}\PY{p}{(}\PY{n}{cost}\PY{p}{,} \PY{n}{feed\PYZus{}dict}\PY{o}{=}\PY{n}{feed\PYZus{}dict\PYZus{}validate}\PY{p}{)}
                     \PY{n}{epoch} \PY{o}{=} \PY{n+nb}{int}\PY{p}{(}\PY{n}{i} \PY{o}{/} \PY{n+nb}{int}\PY{p}{(}\PY{n}{data}\PY{o}{.}\PY{n}{train}\PY{o}{.}\PY{n}{num\PYZus{}examples}\PY{o}{/}\PY{n}{batch\PYZus{}size}\PY{p}{)}\PY{p}{)}
                     
                     \PY{n}{print\PYZus{}progress}\PY{p}{(}\PY{n}{epoch}\PY{p}{,} \PY{n}{feed\PYZus{}dict\PYZus{}train}\PY{p}{,} \PY{n}{feed\PYZus{}dict\PYZus{}validate}\PY{p}{,} \PY{n}{val\PYZus{}loss}\PY{p}{)}
                     
                     \PY{k}{if} \PY{n}{early\PYZus{}stopping}\PY{p}{:}    
                         \PY{k}{if} \PY{n}{val\PYZus{}loss} \PY{o}{\PYZlt{}} \PY{n}{best\PYZus{}val\PYZus{}loss}\PY{p}{:}
                             \PY{n}{best\PYZus{}val\PYZus{}loss} \PY{o}{=} \PY{n}{val\PYZus{}loss}
                             \PY{n}{patience} \PY{o}{=} \PY{l+m+mi}{0}
                         \PY{k}{else}\PY{p}{:}
                             \PY{n}{patience} \PY{o}{+}\PY{o}{=} \PY{l+m+mi}{1}
         
                         \PY{k}{if} \PY{n}{patience} \PY{o}{==} \PY{n}{early\PYZus{}stopping}\PY{p}{:}
                             \PY{k}{break}
         
             \PY{c+c1}{\PYZsh{} Update the total number of iterations performed.}
             \PY{n}{total\PYZus{}iterations} \PY{o}{+}\PY{o}{=} \PY{n}{num\PYZus{}iterations}
         
             \PY{c+c1}{\PYZsh{} Ending time.}
             \PY{n}{end\PYZus{}time} \PY{o}{=} \PY{n}{time}\PY{o}{.}\PY{n}{time}\PY{p}{(}\PY{p}{)}
         
             \PY{c+c1}{\PYZsh{} Difference between start and end\PYZhy{}times.}
             \PY{n}{time\PYZus{}dif} \PY{o}{=} \PY{n}{end\PYZus{}time} \PY{o}{\PYZhy{}} \PY{n}{start\PYZus{}time}
         
             \PY{c+c1}{\PYZsh{} Print the time\PYZhy{}usage.}
             \PY{n+nb}{print}\PY{p}{(}\PY{l+s+s2}{\PYZdq{}}\PY{l+s+s2}{Time elapsed: }\PY{l+s+s2}{\PYZdq{}} \PY{o}{+} \PY{n+nb}{str}\PY{p}{(}\PY{n}{timedelta}\PY{p}{(}\PY{n}{seconds}\PY{o}{=}\PY{n+nb}{int}\PY{p}{(}\PY{n+nb}{round}\PY{p}{(}\PY{n}{time\PYZus{}dif}\PY{p}{)}\PY{p}{)}\PY{p}{)}\PY{p}{)}\PY{p}{)}
\end{Verbatim}


    \subsubsection{Helper-function to plot example
errors}\label{helper-function-to-plot-example-errors}

    Function for plotting examples of images from the test-set that have
been mis-classified.

    \begin{Verbatim}[commandchars=\\\{\}]
{\color{incolor}In [{\color{incolor}41}]:} \PY{k}{def} \PY{n+nf}{plot\PYZus{}example\PYZus{}errors}\PY{p}{(}\PY{n}{cls\PYZus{}pred}\PY{p}{,} \PY{n}{correct}\PY{p}{)}\PY{p}{:}
             \PY{c+c1}{\PYZsh{} cls\PYZus{}pred is an array of the predicted class\PYZhy{}number for}
             \PY{c+c1}{\PYZsh{} all images in the test\PYZhy{}set.}
         
             \PY{c+c1}{\PYZsh{} correct is a boolean array whether the predicted class}
             \PY{c+c1}{\PYZsh{} is equal to the true class for each image in the test\PYZhy{}set.}
         
             \PY{c+c1}{\PYZsh{} Negate the boolean array.}
             \PY{n}{incorrect} \PY{o}{=} \PY{p}{(}\PY{n}{correct} \PY{o}{==} \PY{k+kc}{False}\PY{p}{)}
             
             \PY{c+c1}{\PYZsh{} Get the images from the test\PYZhy{}set that have been}
             \PY{c+c1}{\PYZsh{} incorrectly classified.}
             \PY{n}{images} \PY{o}{=} \PY{n}{data}\PY{o}{.}\PY{n}{valid}\PY{o}{.}\PY{n}{images}\PY{p}{[}\PY{n}{incorrect}\PY{p}{]}
             
             \PY{c+c1}{\PYZsh{} Get the predicted classes for those images.}
             \PY{n}{cls\PYZus{}pred} \PY{o}{=} \PY{n}{cls\PYZus{}pred}\PY{p}{[}\PY{n}{incorrect}\PY{p}{]}
         
             \PY{c+c1}{\PYZsh{} Get the true classes for those images.}
             \PY{n}{cls\PYZus{}true} \PY{o}{=} \PY{n}{data}\PY{o}{.}\PY{n}{valid}\PY{o}{.}\PY{n}{cls}\PY{p}{[}\PY{n}{incorrect}\PY{p}{]}
             
             \PY{c+c1}{\PYZsh{} Plot the first 9 images.}
             \PY{n}{plot\PYZus{}images}\PY{p}{(}\PY{n}{images}\PY{o}{=}\PY{n}{images}\PY{p}{[}\PY{l+m+mi}{0}\PY{p}{:}\PY{l+m+mi}{9}\PY{p}{]}\PY{p}{,}
                         \PY{n}{cls\PYZus{}true}\PY{o}{=}\PY{n}{cls\PYZus{}true}\PY{p}{[}\PY{l+m+mi}{0}\PY{p}{:}\PY{l+m+mi}{9}\PY{p}{]}\PY{p}{,}
                         \PY{n}{cls\PYZus{}pred}\PY{o}{=}\PY{n}{cls\PYZus{}pred}\PY{p}{[}\PY{l+m+mi}{0}\PY{p}{:}\PY{l+m+mi}{9}\PY{p}{]}\PY{p}{)}
\end{Verbatim}


    \subsubsection{Helper-function to plot confusion
matrix}\label{helper-function-to-plot-confusion-matrix}

    \begin{Verbatim}[commandchars=\\\{\}]
{\color{incolor}In [{\color{incolor}42}]:} \PY{k}{def} \PY{n+nf}{plot\PYZus{}confusion\PYZus{}matrix}\PY{p}{(}\PY{n}{cls\PYZus{}pred}\PY{p}{)}\PY{p}{:}
             \PY{c+c1}{\PYZsh{} cls\PYZus{}pred is an array of the predicted class\PYZhy{}number for}
             \PY{c+c1}{\PYZsh{} all images in the test\PYZhy{}set.}
         
             \PY{c+c1}{\PYZsh{} Get the true classifications for the test\PYZhy{}set.}
             \PY{n}{cls\PYZus{}true} \PY{o}{=} \PY{n}{data}\PY{o}{.}\PY{n}{valid}\PY{o}{.}\PY{n}{cls}
             
             \PY{c+c1}{\PYZsh{} Get the confusion matrix using sklearn.}
             \PY{n}{cm} \PY{o}{=} \PY{n}{confusion\PYZus{}matrix}\PY{p}{(}\PY{n}{y\PYZus{}true}\PY{o}{=}\PY{n}{cls\PYZus{}true}\PY{p}{,}
                                   \PY{n}{y\PYZus{}pred}\PY{o}{=}\PY{n}{cls\PYZus{}pred}\PY{p}{)}
         
             \PY{c+c1}{\PYZsh{} Print the confusion matrix as text.}
             \PY{n+nb}{print}\PY{p}{(}\PY{n}{cm}\PY{p}{)}
         
             \PY{c+c1}{\PYZsh{} Plot the confusion matrix as an image.}
             \PY{n}{plt}\PY{o}{.}\PY{n}{matshow}\PY{p}{(}\PY{n}{cm}\PY{p}{)}
         
             \PY{c+c1}{\PYZsh{} Make various adjustments to the plot.}
             \PY{n}{plt}\PY{o}{.}\PY{n}{colorbar}\PY{p}{(}\PY{p}{)}
             \PY{n}{tick\PYZus{}marks} \PY{o}{=} \PY{n}{np}\PY{o}{.}\PY{n}{arange}\PY{p}{(}\PY{n}{num\PYZus{}classes}\PY{p}{)}
             \PY{n}{plt}\PY{o}{.}\PY{n}{xticks}\PY{p}{(}\PY{n}{tick\PYZus{}marks}\PY{p}{,} \PY{n+nb}{range}\PY{p}{(}\PY{n}{num\PYZus{}classes}\PY{p}{)}\PY{p}{)}
             \PY{n}{plt}\PY{o}{.}\PY{n}{yticks}\PY{p}{(}\PY{n}{tick\PYZus{}marks}\PY{p}{,} \PY{n+nb}{range}\PY{p}{(}\PY{n}{num\PYZus{}classes}\PY{p}{)}\PY{p}{)}
             \PY{n}{plt}\PY{o}{.}\PY{n}{xlabel}\PY{p}{(}\PY{l+s+s1}{\PYZsq{}}\PY{l+s+s1}{Predicted}\PY{l+s+s1}{\PYZsq{}}\PY{p}{)}
             \PY{n}{plt}\PY{o}{.}\PY{n}{ylabel}\PY{p}{(}\PY{l+s+s1}{\PYZsq{}}\PY{l+s+s1}{True}\PY{l+s+s1}{\PYZsq{}}\PY{p}{)}
         
             \PY{c+c1}{\PYZsh{} Ensure the plot is shown correctly with multiple plots}
             \PY{c+c1}{\PYZsh{} in a single Notebook cell.}
             \PY{n}{plt}\PY{o}{.}\PY{n}{show}\PY{p}{(}\PY{p}{)}
\end{Verbatim}


    \subsubsection{Helper-function for showing the
performance}\label{helper-function-for-showing-the-performance}

    Function for printing the classification accuracy on the test-set.

It takes a while to compute the classification for all the images in the
test-set, that's why the results are re-used by calling the above
functions directly from this function, so the classifications don't have
to be recalculated by each function.

Note that this function can use a lot of computer memory, which is why
the test-set is split into smaller batches. If you have little RAM in
your computer and it crashes, then you can try and lower the batch-size.

    \begin{Verbatim}[commandchars=\\\{\}]
{\color{incolor}In [{\color{incolor}43}]:} \PY{k}{def} \PY{n+nf}{print\PYZus{}validation\PYZus{}accuracy}\PY{p}{(}\PY{n}{show\PYZus{}example\PYZus{}errors}\PY{o}{=}\PY{k+kc}{False}\PY{p}{,}
                                 \PY{n}{show\PYZus{}confusion\PYZus{}matrix}\PY{o}{=}\PY{k+kc}{False}\PY{p}{)}\PY{p}{:}
         
             \PY{c+c1}{\PYZsh{} Number of images in the test\PYZhy{}set.}
             \PY{n}{num\PYZus{}test} \PY{o}{=} \PY{n+nb}{len}\PY{p}{(}\PY{n}{data}\PY{o}{.}\PY{n}{valid}\PY{o}{.}\PY{n}{images}\PY{p}{)}
         
             \PY{c+c1}{\PYZsh{} Allocate an array for the predicted classes which}
             \PY{c+c1}{\PYZsh{} will be calculated in batches and filled into this array.}
             \PY{n}{cls\PYZus{}pred} \PY{o}{=} \PY{n}{np}\PY{o}{.}\PY{n}{zeros}\PY{p}{(}\PY{n}{shape}\PY{o}{=}\PY{n}{num\PYZus{}test}\PY{p}{,} \PY{n}{dtype}\PY{o}{=}\PY{n}{np}\PY{o}{.}\PY{n}{int}\PY{p}{)}
         
             \PY{c+c1}{\PYZsh{} Now calculate the predicted classes for the batches.}
             \PY{c+c1}{\PYZsh{} We will just iterate through all the batches.}
             \PY{c+c1}{\PYZsh{} There might be a more clever and Pythonic way of doing this.}
         
             \PY{c+c1}{\PYZsh{} The starting index for the next batch is denoted i.}
             \PY{n}{i} \PY{o}{=} \PY{l+m+mi}{0}
         
             \PY{k}{while} \PY{n}{i} \PY{o}{\PYZlt{}} \PY{n}{num\PYZus{}test}\PY{p}{:}
                 \PY{c+c1}{\PYZsh{} The ending index for the next batch is denoted j.}
                 \PY{n}{j} \PY{o}{=} \PY{n+nb}{min}\PY{p}{(}\PY{n}{i} \PY{o}{+} \PY{n}{batch\PYZus{}size}\PY{p}{,} \PY{n}{num\PYZus{}test}\PY{p}{)}
         
                 \PY{c+c1}{\PYZsh{} Get the images from the test\PYZhy{}set between index i and j.}
                 \PY{n}{images} \PY{o}{=} \PY{n}{data}\PY{o}{.}\PY{n}{valid}\PY{o}{.}\PY{n}{images}\PY{p}{[}\PY{n}{i}\PY{p}{:}\PY{n}{j}\PY{p}{,} \PY{p}{:}\PY{p}{]}\PY{o}{.}\PY{n}{reshape}\PY{p}{(}\PY{n}{batch\PYZus{}size}\PY{p}{,} \PY{n}{img\PYZus{}size\PYZus{}flat}\PY{p}{)}
                 
         
                 \PY{c+c1}{\PYZsh{} Get the associated labels.}
                 \PY{n}{labels} \PY{o}{=} \PY{n}{data}\PY{o}{.}\PY{n}{valid}\PY{o}{.}\PY{n}{labels}\PY{p}{[}\PY{n}{i}\PY{p}{:}\PY{n}{j}\PY{p}{,} \PY{p}{:}\PY{p}{]}
         
                 \PY{c+c1}{\PYZsh{} Create a feed\PYZhy{}dict with these images and labels.}
                 \PY{n}{feed\PYZus{}dict} \PY{o}{=} \PY{p}{\PYZob{}}\PY{n}{x}\PY{p}{:} \PY{n}{images}\PY{p}{,}
                              \PY{n}{y\PYZus{}true}\PY{p}{:} \PY{n}{labels}\PY{p}{\PYZcb{}}
         
                 \PY{c+c1}{\PYZsh{} Calculate the predicted class using TensorFlow.}
                 \PY{n}{cls\PYZus{}pred}\PY{p}{[}\PY{n}{i}\PY{p}{:}\PY{n}{j}\PY{p}{]} \PY{o}{=} \PY{n}{session}\PY{o}{.}\PY{n}{run}\PY{p}{(}\PY{n}{y\PYZus{}pred\PYZus{}cls}\PY{p}{,} \PY{n}{feed\PYZus{}dict}\PY{o}{=}\PY{n}{feed\PYZus{}dict}\PY{p}{)}
         
                 \PY{c+c1}{\PYZsh{} Set the start\PYZhy{}index for the next batch to the}
                 \PY{c+c1}{\PYZsh{} end\PYZhy{}index of the current batch.}
                 \PY{n}{i} \PY{o}{=} \PY{n}{j}
         
             \PY{n}{cls\PYZus{}true} \PY{o}{=} \PY{n}{np}\PY{o}{.}\PY{n}{array}\PY{p}{(}\PY{n}{data}\PY{o}{.}\PY{n}{valid}\PY{o}{.}\PY{n}{cls}\PY{p}{)}
             \PY{n}{cls\PYZus{}pred} \PY{o}{=} \PY{n}{np}\PY{o}{.}\PY{n}{array}\PY{p}{(}\PY{p}{[}\PY{n}{classes}\PY{p}{[}\PY{n}{x}\PY{p}{]} \PY{k}{for} \PY{n}{x} \PY{o+ow}{in} \PY{n}{cls\PYZus{}pred}\PY{p}{]}\PY{p}{)} 
         
             \PY{c+c1}{\PYZsh{} Create a boolean array whether each image is correctly classified.}
             \PY{n}{correct} \PY{o}{=} \PY{p}{(}\PY{n}{cls\PYZus{}true} \PY{o}{==} \PY{n}{cls\PYZus{}pred}\PY{p}{)}
         
             \PY{c+c1}{\PYZsh{} Calculate the number of correctly classified images.}
             \PY{c+c1}{\PYZsh{} When summing a boolean array, False means 0 and True means 1.}
             \PY{n}{correct\PYZus{}sum} \PY{o}{=} \PY{n}{correct}\PY{o}{.}\PY{n}{sum}\PY{p}{(}\PY{p}{)}
         
             \PY{c+c1}{\PYZsh{} Classification accuracy is the number of correctly classified}
             \PY{c+c1}{\PYZsh{} images divided by the total number of images in the test\PYZhy{}set.}
             \PY{n}{acc} \PY{o}{=} \PY{n+nb}{float}\PY{p}{(}\PY{n}{correct\PYZus{}sum}\PY{p}{)} \PY{o}{/} \PY{n}{num\PYZus{}test}
         
             \PY{c+c1}{\PYZsh{} Print the accuracy.}
             \PY{n}{msg} \PY{o}{=} \PY{l+s+s2}{\PYZdq{}}\PY{l+s+s2}{Accuracy on Test\PYZhy{}Set: }\PY{l+s+si}{\PYZob{}0:.1\PYZpc{}\PYZcb{}}\PY{l+s+s2}{ (}\PY{l+s+si}{\PYZob{}1\PYZcb{}}\PY{l+s+s2}{ / }\PY{l+s+si}{\PYZob{}2\PYZcb{}}\PY{l+s+s2}{)}\PY{l+s+s2}{\PYZdq{}}
             \PY{n+nb}{print}\PY{p}{(}\PY{n}{msg}\PY{o}{.}\PY{n}{format}\PY{p}{(}\PY{n}{acc}\PY{p}{,} \PY{n}{correct\PYZus{}sum}\PY{p}{,} \PY{n}{num\PYZus{}test}\PY{p}{)}\PY{p}{)}
         
             \PY{c+c1}{\PYZsh{} Plot some examples of mis\PYZhy{}classifications, if desired.}
             \PY{k}{if} \PY{n}{show\PYZus{}example\PYZus{}errors}\PY{p}{:}
                 \PY{n+nb}{print}\PY{p}{(}\PY{l+s+s2}{\PYZdq{}}\PY{l+s+s2}{Example errors:}\PY{l+s+s2}{\PYZdq{}}\PY{p}{)}
                 \PY{n}{plot\PYZus{}example\PYZus{}errors}\PY{p}{(}\PY{n}{cls\PYZus{}pred}\PY{o}{=}\PY{n}{cls\PYZus{}pred}\PY{p}{,} \PY{n}{correct}\PY{o}{=}\PY{n}{correct}\PY{p}{)}
         
             \PY{c+c1}{\PYZsh{} Plot the confusion matrix, if desired.}
             \PY{k}{if} \PY{n}{show\PYZus{}confusion\PYZus{}matrix}\PY{p}{:}
                 \PY{n+nb}{print}\PY{p}{(}\PY{l+s+s2}{\PYZdq{}}\PY{l+s+s2}{Confusion Matrix:}\PY{l+s+s2}{\PYZdq{}}\PY{p}{)}
                 \PY{n}{plot\PYZus{}confusion\PYZus{}matrix}\PY{p}{(}\PY{n}{cls\PYZus{}pred}\PY{o}{=}\PY{n}{cls\PYZus{}pred}\PY{p}{)}
\end{Verbatim}


    \subsection{Performance after 1 optimization
iteration}\label{performance-after-1-optimization-iteration}

    \begin{Verbatim}[commandchars=\\\{\}]
{\color{incolor}In [{\color{incolor}44}]:} \PY{n}{optimize}\PY{p}{(}\PY{n}{num\PYZus{}iterations}\PY{o}{=}\PY{l+m+mi}{1}\PY{p}{)}
         \PY{n}{print\PYZus{}validation\PYZus{}accuracy}\PY{p}{(}\PY{p}{)}
\end{Verbatim}


    \begin{Verbatim}[commandchars=\\\{\}]
Epoch 1 --- Training Accuracy:  53.1\%, Validation Accuracy:  53.1\%, Validation Loss: 0.699
Time elapsed: 0:00:04
Accuracy on Test-Set: 49.6\% (1986 / 4000)

    \end{Verbatim}

    \subsection{Performance after 100 optimization
iterations}\label{performance-after-100-optimization-iterations}

After 100 optimization iterations, the model should have significantly
improved its classification accuracy.

    \begin{Verbatim}[commandchars=\\\{\}]
{\color{incolor}In [{\color{incolor}45}]:} \PY{n}{optimize}\PY{p}{(}\PY{n}{num\PYZus{}iterations}\PY{o}{=}\PY{l+m+mi}{99}\PY{p}{)}  \PY{c+c1}{\PYZsh{} We already performed 1 iteration above.}
\end{Verbatim}


    \begin{Verbatim}[commandchars=\\\{\}]
Time elapsed: 0:02:28

    \end{Verbatim}

    \begin{Verbatim}[commandchars=\\\{\}]
{\color{incolor}In [{\color{incolor}46}]:} \PY{n}{print\PYZus{}validation\PYZus{}accuracy}\PY{p}{(}\PY{n}{show\PYZus{}example\PYZus{}errors}\PY{o}{=}\PY{k+kc}{True}\PY{p}{)}
\end{Verbatim}


    \begin{Verbatim}[commandchars=\\\{\}]
Accuracy on Test-Set: 50.7\% (2028 / 4000)
Example errors:

    \end{Verbatim}

    \begin{center}
    \adjustimage{max size={0.9\linewidth}{0.9\paperheight}}{output_94_1.png}
    \end{center}
    { \hspace*{\fill} \\}
    
    \subsection{Performance after 1000 optimization
iterations}\label{performance-after-1000-optimization-iterations}

    \begin{Verbatim}[commandchars=\\\{\}]
{\color{incolor}In [{\color{incolor}47}]:} \PY{n}{optimize}\PY{p}{(}\PY{n}{num\PYZus{}iterations}\PY{o}{=}\PY{l+m+mi}{900}\PY{p}{)}  \PY{c+c1}{\PYZsh{} We performed 100 iterations above.}
\end{Verbatim}


    \begin{Verbatim}[commandchars=\\\{\}]
Epoch 2 --- Training Accuracy:  62.5\%, Validation Accuracy:  53.1\%, Validation Loss: 0.686
Time elapsed: 0:17:39

    \end{Verbatim}

    \begin{Verbatim}[commandchars=\\\{\}]
{\color{incolor}In [{\color{incolor}48}]:} \PY{n}{print\PYZus{}validation\PYZus{}accuracy}\PY{p}{(}\PY{n}{show\PYZus{}example\PYZus{}errors}\PY{o}{=}\PY{k+kc}{True}\PY{p}{)}
\end{Verbatim}


    \begin{Verbatim}[commandchars=\\\{\}]
Accuracy on Test-Set: 67.9\% (2715 / 4000)
Example errors:

    \end{Verbatim}

    \begin{center}
    \adjustimage{max size={0.9\linewidth}{0.9\paperheight}}{output_97_1.png}
    \end{center}
    { \hspace*{\fill} \\}
    
    \subsection{Performance after 10,000 optimization
iterations}\label{performance-after-10000-optimization-iterations}

    \begin{Verbatim}[commandchars=\\\{\}]
{\color{incolor}In [{\color{incolor}49}]:} \PY{n}{optimize}\PY{p}{(}\PY{n}{num\PYZus{}iterations}\PY{o}{=}\PY{l+m+mi}{9000}\PY{p}{)} \PY{c+c1}{\PYZsh{} We performed 1000 iterations above.}
\end{Verbatim}


    \begin{Verbatim}[commandchars=\\\{\}]
Epoch 3 --- Training Accuracy:  65.6\%, Validation Accuracy:  65.6\%, Validation Loss: 0.606
Epoch 4 --- Training Accuracy:  71.9\%, Validation Accuracy:  81.2\%, Validation Loss: 0.426
Epoch 5 --- Training Accuracy:  78.1\%, Validation Accuracy:  75.0\%, Validation Loss: 0.527
Epoch 6 --- Training Accuracy:  81.2\%, Validation Accuracy:  78.1\%, Validation Loss: 0.473
Epoch 7 --- Training Accuracy:  81.2\%, Validation Accuracy:  81.2\%, Validation Loss: 0.461
Epoch 8 --- Training Accuracy:  84.4\%, Validation Accuracy:  59.4\%, Validation Loss: 0.734
Epoch 9 --- Training Accuracy:  84.4\%, Validation Accuracy:  75.0\%, Validation Loss: 0.487
Epoch 10 --- Training Accuracy:  90.6\%, Validation Accuracy:  68.8\%, Validation Loss: 0.607
Epoch 11 --- Training Accuracy:  90.6\%, Validation Accuracy:  93.8\%, Validation Loss: 0.239
Epoch 12 --- Training Accuracy:  90.6\%, Validation Accuracy:  81.2\%, Validation Loss: 0.353
Epoch 13 --- Training Accuracy:  90.6\%, Validation Accuracy:  90.6\%, Validation Loss: 0.264
Epoch 14 --- Training Accuracy:  90.6\%, Validation Accuracy:  84.4\%, Validation Loss: 0.415
Epoch 15 --- Training Accuracy:  93.8\%, Validation Accuracy:  65.6\%, Validation Loss: 0.717
Epoch 16 --- Training Accuracy:  93.8\%, Validation Accuracy:  68.8\%, Validation Loss: 0.653
Time elapsed: 3:33:24

    \end{Verbatim}

    \begin{Verbatim}[commandchars=\\\{\}]
{\color{incolor}In [{\color{incolor}50}]:} \PY{n}{print\PYZus{}validation\PYZus{}accuracy}\PY{p}{(}\PY{n}{show\PYZus{}example\PYZus{}errors}\PY{o}{=}\PY{k+kc}{True}\PY{p}{,} \PY{n}{show\PYZus{}confusion\PYZus{}matrix}\PY{o}{=}\PY{k+kc}{True}\PY{p}{)}
\end{Verbatim}


    \begin{Verbatim}[commandchars=\\\{\}]
Accuracy on Test-Set: 79.5\% (3181 / 4000)
Example errors:

    \end{Verbatim}

    \begin{center}
    \adjustimage{max size={0.9\linewidth}{0.9\paperheight}}{output_100_1.png}
    \end{center}
    { \hspace*{\fill} \\}
    
    \begin{Verbatim}[commandchars=\\\{\}]
Confusion Matrix:
[[1670  344]
 [ 475 1511]]

    \end{Verbatim}

    \begin{center}
    \adjustimage{max size={0.9\linewidth}{0.9\paperheight}}{output_100_3.png}
    \end{center}
    { \hspace*{\fill} \\}
    
    \subsection{Test on Sample Image}\label{test-on-sample-image}

    \begin{Verbatim}[commandchars=\\\{\}]
{\color{incolor}In [{\color{incolor}56}]:} \PY{n}{plt}\PY{o}{.}\PY{n}{axis}\PY{p}{(}\PY{l+s+s1}{\PYZsq{}}\PY{l+s+s1}{off}\PY{l+s+s1}{\PYZsq{}}\PY{p}{)}
         
         \PY{n}{test\PYZus{}cat} \PY{o}{=} \PY{n}{cv2}\PY{o}{.}\PY{n}{imread}\PY{p}{(}\PY{l+s+s1}{\PYZsq{}}\PY{l+s+s1}{cat.jpg}\PY{l+s+s1}{\PYZsq{}}\PY{p}{)}
         \PY{n}{test\PYZus{}cat} \PY{o}{=} \PY{n}{cv2}\PY{o}{.}\PY{n}{resize}\PY{p}{(}\PY{n}{test\PYZus{}cat}\PY{p}{,} \PY{p}{(}\PY{n}{img\PYZus{}size}\PY{p}{,} \PY{n}{img\PYZus{}size}\PY{p}{)}\PY{p}{,} \PY{n}{cv2}\PY{o}{.}\PY{n}{INTER\PYZus{}LINEAR}\PY{p}{)} \PY{o}{/} \PY{l+m+mi}{255}
         
         \PY{n}{preview\PYZus{}cat} \PY{o}{=} \PY{n}{plt}\PY{o}{.}\PY{n}{imshow}\PY{p}{(}\PY{n}{test\PYZus{}cat}\PY{o}{.}\PY{n}{reshape}\PY{p}{(}\PY{n}{img\PYZus{}size}\PY{p}{,} \PY{n}{img\PYZus{}size}\PY{p}{,} \PY{n}{num\PYZus{}channels}\PY{p}{)}\PY{p}{)}
\end{Verbatim}


    \begin{center}
    \adjustimage{max size={0.9\linewidth}{0.9\paperheight}}{output_102_0.png}
    \end{center}
    { \hspace*{\fill} \\}
    
    \begin{Verbatim}[commandchars=\\\{\}]
{\color{incolor}In [{\color{incolor}57}]:} \PY{n}{test\PYZus{}dog} \PY{o}{=} \PY{n}{cv2}\PY{o}{.}\PY{n}{imread}\PY{p}{(}\PY{l+s+s1}{\PYZsq{}}\PY{l+s+s1}{lucy.jpg}\PY{l+s+s1}{\PYZsq{}}\PY{p}{)}
         \PY{n}{test\PYZus{}dog} \PY{o}{=} \PY{n}{cv2}\PY{o}{.}\PY{n}{resize}\PY{p}{(}\PY{n}{test\PYZus{}dog}\PY{p}{,} \PY{p}{(}\PY{n}{img\PYZus{}size}\PY{p}{,} \PY{n}{img\PYZus{}size}\PY{p}{)}\PY{p}{,} \PY{n}{cv2}\PY{o}{.}\PY{n}{INTER\PYZus{}LINEAR}\PY{p}{)} \PY{o}{/} \PY{l+m+mi}{255}
         
         \PY{n}{preview\PYZus{}dog} \PY{o}{=} \PY{n}{plt}\PY{o}{.}\PY{n}{imshow}\PY{p}{(}\PY{n}{test\PYZus{}dog}\PY{o}{.}\PY{n}{reshape}\PY{p}{(}\PY{n}{img\PYZus{}size}\PY{p}{,} \PY{n}{img\PYZus{}size}\PY{p}{,} \PY{n}{num\PYZus{}channels}\PY{p}{)}\PY{p}{)}
\end{Verbatim}


    \begin{center}
    \adjustimage{max size={0.9\linewidth}{0.9\paperheight}}{output_103_0.png}
    \end{center}
    { \hspace*{\fill} \\}
    
    \begin{Verbatim}[commandchars=\\\{\}]
{\color{incolor}In [{\color{incolor}58}]:} \PY{k}{def} \PY{n+nf}{sample\PYZus{}prediction}\PY{p}{(}\PY{n}{test\PYZus{}im}\PY{p}{)}\PY{p}{:}
             
             \PY{n}{feed\PYZus{}dict\PYZus{}test} \PY{o}{=} \PY{p}{\PYZob{}}
                 \PY{n}{x}\PY{p}{:} \PY{n}{test\PYZus{}im}\PY{o}{.}\PY{n}{reshape}\PY{p}{(}\PY{l+m+mi}{1}\PY{p}{,} \PY{n}{img\PYZus{}size\PYZus{}flat}\PY{p}{)}\PY{p}{,}
                 \PY{n}{y\PYZus{}true}\PY{p}{:} \PY{n}{np}\PY{o}{.}\PY{n}{array}\PY{p}{(}\PY{p}{[}\PY{p}{[}\PY{l+m+mi}{1}\PY{p}{,} \PY{l+m+mi}{0}\PY{p}{]}\PY{p}{]}\PY{p}{)}
             \PY{p}{\PYZcb{}}
         
             \PY{n}{test\PYZus{}pred} \PY{o}{=} \PY{n}{session}\PY{o}{.}\PY{n}{run}\PY{p}{(}\PY{n}{y\PYZus{}pred\PYZus{}cls}\PY{p}{,} \PY{n}{feed\PYZus{}dict}\PY{o}{=}\PY{n}{feed\PYZus{}dict\PYZus{}test}\PY{p}{)}
             \PY{k}{return} \PY{n}{classes}\PY{p}{[}\PY{n}{test\PYZus{}pred}\PY{p}{[}\PY{l+m+mi}{0}\PY{p}{]}\PY{p}{]}
         
         \PY{n+nb}{print}\PY{p}{(}\PY{l+s+s2}{\PYZdq{}}\PY{l+s+s2}{Predicted class for test\PYZus{}cat: }\PY{l+s+si}{\PYZob{}\PYZcb{}}\PY{l+s+s2}{\PYZdq{}}\PY{o}{.}\PY{n}{format}\PY{p}{(}\PY{n}{sample\PYZus{}prediction}\PY{p}{(}\PY{n}{test\PYZus{}cat}\PY{p}{)}\PY{p}{)}\PY{p}{)}
         \PY{n+nb}{print}\PY{p}{(}\PY{l+s+s2}{\PYZdq{}}\PY{l+s+s2}{Predicted class for test\PYZus{}dog: }\PY{l+s+si}{\PYZob{}\PYZcb{}}\PY{l+s+s2}{\PYZdq{}}\PY{o}{.}\PY{n}{format}\PY{p}{(}\PY{n}{sample\PYZus{}prediction}\PY{p}{(}\PY{n}{test\PYZus{}dog}\PY{p}{)}\PY{p}{)}\PY{p}{)}
\end{Verbatim}


    \begin{Verbatim}[commandchars=\\\{\}]
Predicted class for test\_cat: dogs
Predicted class for test\_dog: cats

    \end{Verbatim}

    \subsection{Visualization of Weights and
Layers}\label{visualization-of-weights-and-layers}

In trying to understand why the convolutional neural network can
recognize images, we will now visualize the weights of the convolutional
filters and the resulting output images.

    \subsubsection{Helper-function for plotting convolutional
weights}\label{helper-function-for-plotting-convolutional-weights}

    \begin{Verbatim}[commandchars=\\\{\}]
{\color{incolor}In [{\color{incolor}59}]:} \PY{k}{def} \PY{n+nf}{plot\PYZus{}conv\PYZus{}weights}\PY{p}{(}\PY{n}{weights}\PY{p}{,} \PY{n}{input\PYZus{}channel}\PY{o}{=}\PY{l+m+mi}{0}\PY{p}{)}\PY{p}{:}
             \PY{c+c1}{\PYZsh{} Assume weights are TensorFlow ops for 4\PYZhy{}dim variables}
             \PY{c+c1}{\PYZsh{} e.g. weights\PYZus{}conv1 or weights\PYZus{}conv2.}
             
             \PY{c+c1}{\PYZsh{} Retrieve the values of the weight\PYZhy{}variables from TensorFlow.}
             \PY{c+c1}{\PYZsh{} A feed\PYZhy{}dict is not necessary because nothing is calculated.}
             \PY{n}{w} \PY{o}{=} \PY{n}{session}\PY{o}{.}\PY{n}{run}\PY{p}{(}\PY{n}{weights}\PY{p}{)}
         
             \PY{c+c1}{\PYZsh{} Get the lowest and highest values for the weights.}
             \PY{c+c1}{\PYZsh{} This is used to correct the colour intensity across}
             \PY{c+c1}{\PYZsh{} the images so they can be compared with each other.}
             \PY{n}{w\PYZus{}min} \PY{o}{=} \PY{n}{np}\PY{o}{.}\PY{n}{min}\PY{p}{(}\PY{n}{w}\PY{p}{)}
             \PY{n}{w\PYZus{}max} \PY{o}{=} \PY{n}{np}\PY{o}{.}\PY{n}{max}\PY{p}{(}\PY{n}{w}\PY{p}{)}
         
             \PY{c+c1}{\PYZsh{} Number of filters used in the conv. layer.}
             \PY{n}{num\PYZus{}filters} \PY{o}{=} \PY{n}{w}\PY{o}{.}\PY{n}{shape}\PY{p}{[}\PY{l+m+mi}{3}\PY{p}{]}
         
             \PY{c+c1}{\PYZsh{} Number of grids to plot.}
             \PY{c+c1}{\PYZsh{} Rounded\PYZhy{}up, square\PYZhy{}root of the number of filters.}
             \PY{n}{num\PYZus{}grids} \PY{o}{=} \PY{n}{math}\PY{o}{.}\PY{n}{ceil}\PY{p}{(}\PY{n}{math}\PY{o}{.}\PY{n}{sqrt}\PY{p}{(}\PY{n}{num\PYZus{}filters}\PY{p}{)}\PY{p}{)}
             
             \PY{c+c1}{\PYZsh{} Create figure with a grid of sub\PYZhy{}plots.}
             \PY{n}{fig}\PY{p}{,} \PY{n}{axes} \PY{o}{=} \PY{n}{plt}\PY{o}{.}\PY{n}{subplots}\PY{p}{(}\PY{n}{num\PYZus{}grids}\PY{p}{,} \PY{n}{num\PYZus{}grids}\PY{p}{)}
         
             \PY{c+c1}{\PYZsh{} Plot all the filter\PYZhy{}weights.}
             \PY{k}{for} \PY{n}{i}\PY{p}{,} \PY{n}{ax} \PY{o+ow}{in} \PY{n+nb}{enumerate}\PY{p}{(}\PY{n}{axes}\PY{o}{.}\PY{n}{flat}\PY{p}{)}\PY{p}{:}
                 \PY{c+c1}{\PYZsh{} Only plot the valid filter\PYZhy{}weights.}
                 \PY{k}{if} \PY{n}{i}\PY{o}{\PYZlt{}}\PY{n}{num\PYZus{}filters}\PY{p}{:}
                     \PY{c+c1}{\PYZsh{} Get the weights for the i\PYZsq{}th filter of the input channel.}
                     \PY{c+c1}{\PYZsh{} See new\PYZus{}conv\PYZus{}layer() for details on the format}
                     \PY{c+c1}{\PYZsh{} of this 4\PYZhy{}dim tensor.}
                     \PY{n}{img} \PY{o}{=} \PY{n}{w}\PY{p}{[}\PY{p}{:}\PY{p}{,} \PY{p}{:}\PY{p}{,} \PY{n}{input\PYZus{}channel}\PY{p}{,} \PY{n}{i}\PY{p}{]}
         
                     \PY{c+c1}{\PYZsh{} Plot image.}
                     \PY{n}{ax}\PY{o}{.}\PY{n}{imshow}\PY{p}{(}\PY{n}{img}\PY{p}{,} \PY{n}{vmin}\PY{o}{=}\PY{n}{w\PYZus{}min}\PY{p}{,} \PY{n}{vmax}\PY{o}{=}\PY{n}{w\PYZus{}max}\PY{p}{,}
                               \PY{n}{interpolation}\PY{o}{=}\PY{l+s+s1}{\PYZsq{}}\PY{l+s+s1}{nearest}\PY{l+s+s1}{\PYZsq{}}\PY{p}{,} \PY{n}{cmap}\PY{o}{=}\PY{l+s+s1}{\PYZsq{}}\PY{l+s+s1}{seismic}\PY{l+s+s1}{\PYZsq{}}\PY{p}{)}
                 
                 \PY{c+c1}{\PYZsh{} Remove ticks from the plot.}
                 \PY{n}{ax}\PY{o}{.}\PY{n}{set\PYZus{}xticks}\PY{p}{(}\PY{p}{[}\PY{p}{]}\PY{p}{)}
                 \PY{n}{ax}\PY{o}{.}\PY{n}{set\PYZus{}yticks}\PY{p}{(}\PY{p}{[}\PY{p}{]}\PY{p}{)}
             
             \PY{c+c1}{\PYZsh{} Ensure the plot is shown correctly with multiple plots}
             \PY{c+c1}{\PYZsh{} in a single Notebook cell.}
             \PY{n}{plt}\PY{o}{.}\PY{n}{show}\PY{p}{(}\PY{p}{)}
\end{Verbatim}


    \subsubsection{Helper-function for plotting the output of a
convolutional
layer}\label{helper-function-for-plotting-the-output-of-a-convolutional-layer}

    \begin{Verbatim}[commandchars=\\\{\}]
{\color{incolor}In [{\color{incolor}60}]:} \PY{k}{def} \PY{n+nf}{plot\PYZus{}conv\PYZus{}layer}\PY{p}{(}\PY{n}{layer}\PY{p}{,} \PY{n}{image}\PY{p}{)}\PY{p}{:}
             \PY{c+c1}{\PYZsh{} Assume layer is a TensorFlow op that outputs a 4\PYZhy{}dim tensor}
             \PY{c+c1}{\PYZsh{} which is the output of a convolutional layer,}
             \PY{c+c1}{\PYZsh{} e.g. layer\PYZus{}conv1 or layer\PYZus{}conv2.}
             
             \PY{n}{image} \PY{o}{=} \PY{n}{image}\PY{o}{.}\PY{n}{reshape}\PY{p}{(}\PY{n}{img\PYZus{}size\PYZus{}flat}\PY{p}{)}
         
             \PY{c+c1}{\PYZsh{} Create a feed\PYZhy{}dict containing just one image.}
             \PY{c+c1}{\PYZsh{} Note that we don\PYZsq{}t need to feed y\PYZus{}true because it is}
             \PY{c+c1}{\PYZsh{} not used in this calculation.}
             \PY{n}{feed\PYZus{}dict} \PY{o}{=} \PY{p}{\PYZob{}}\PY{n}{x}\PY{p}{:} \PY{p}{[}\PY{n}{image}\PY{p}{]}\PY{p}{\PYZcb{}}
         
             \PY{c+c1}{\PYZsh{} Calculate and retrieve the output values of the layer}
             \PY{c+c1}{\PYZsh{} when inputting that image.}
             \PY{n}{values} \PY{o}{=} \PY{n}{session}\PY{o}{.}\PY{n}{run}\PY{p}{(}\PY{n}{layer}\PY{p}{,} \PY{n}{feed\PYZus{}dict}\PY{o}{=}\PY{n}{feed\PYZus{}dict}\PY{p}{)}
         
             \PY{c+c1}{\PYZsh{} Number of filters used in the conv. layer.}
             \PY{n}{num\PYZus{}filters} \PY{o}{=} \PY{n}{values}\PY{o}{.}\PY{n}{shape}\PY{p}{[}\PY{l+m+mi}{3}\PY{p}{]}
         
             \PY{c+c1}{\PYZsh{} Number of grids to plot.}
             \PY{c+c1}{\PYZsh{} Rounded\PYZhy{}up, square\PYZhy{}root of the number of filters.}
             \PY{n}{num\PYZus{}grids} \PY{o}{=} \PY{n}{math}\PY{o}{.}\PY{n}{ceil}\PY{p}{(}\PY{n}{math}\PY{o}{.}\PY{n}{sqrt}\PY{p}{(}\PY{n}{num\PYZus{}filters}\PY{p}{)}\PY{p}{)}
             
             \PY{c+c1}{\PYZsh{} Create figure with a grid of sub\PYZhy{}plots.}
             \PY{n}{fig}\PY{p}{,} \PY{n}{axes} \PY{o}{=} \PY{n}{plt}\PY{o}{.}\PY{n}{subplots}\PY{p}{(}\PY{n}{num\PYZus{}grids}\PY{p}{,} \PY{n}{num\PYZus{}grids}\PY{p}{)}
         
             \PY{c+c1}{\PYZsh{} Plot the output images of all the filters.}
             \PY{k}{for} \PY{n}{i}\PY{p}{,} \PY{n}{ax} \PY{o+ow}{in} \PY{n+nb}{enumerate}\PY{p}{(}\PY{n}{axes}\PY{o}{.}\PY{n}{flat}\PY{p}{)}\PY{p}{:}
                 \PY{c+c1}{\PYZsh{} Only plot the images for valid filters.}
                 \PY{k}{if} \PY{n}{i}\PY{o}{\PYZlt{}}\PY{n}{num\PYZus{}filters}\PY{p}{:}
                     \PY{c+c1}{\PYZsh{} Get the output image of using the i\PYZsq{}th filter.}
                     \PY{c+c1}{\PYZsh{} See new\PYZus{}conv\PYZus{}layer() for details on the format}
                     \PY{c+c1}{\PYZsh{} of this 4\PYZhy{}dim tensor.}
                     \PY{n}{img} \PY{o}{=} \PY{n}{values}\PY{p}{[}\PY{l+m+mi}{0}\PY{p}{,} \PY{p}{:}\PY{p}{,} \PY{p}{:}\PY{p}{,} \PY{n}{i}\PY{p}{]}
         
                     \PY{c+c1}{\PYZsh{} Plot image.}
                     \PY{n}{ax}\PY{o}{.}\PY{n}{imshow}\PY{p}{(}\PY{n}{img}\PY{p}{,} \PY{n}{interpolation}\PY{o}{=}\PY{l+s+s1}{\PYZsq{}}\PY{l+s+s1}{nearest}\PY{l+s+s1}{\PYZsq{}}\PY{p}{,} \PY{n}{cmap}\PY{o}{=}\PY{l+s+s1}{\PYZsq{}}\PY{l+s+s1}{binary}\PY{l+s+s1}{\PYZsq{}}\PY{p}{)}
                 
                 \PY{c+c1}{\PYZsh{} Remove ticks from the plot.}
                 \PY{n}{ax}\PY{o}{.}\PY{n}{set\PYZus{}xticks}\PY{p}{(}\PY{p}{[}\PY{p}{]}\PY{p}{)}
                 \PY{n}{ax}\PY{o}{.}\PY{n}{set\PYZus{}yticks}\PY{p}{(}\PY{p}{[}\PY{p}{]}\PY{p}{)}
             
             \PY{c+c1}{\PYZsh{} Ensure the plot is shown correctly with multiple plots}
             \PY{c+c1}{\PYZsh{} in a single Notebook cell.}
             \PY{n}{plt}\PY{o}{.}\PY{n}{show}\PY{p}{(}\PY{p}{)}
\end{Verbatim}


    \subsubsection{Input Images}\label{input-images}

    Helper-function for plotting an image.

    \begin{Verbatim}[commandchars=\\\{\}]
{\color{incolor}In [{\color{incolor}61}]:} \PY{k}{def} \PY{n+nf}{plot\PYZus{}image}\PY{p}{(}\PY{n}{image}\PY{p}{)}\PY{p}{:}
             \PY{n}{plt}\PY{o}{.}\PY{n}{imshow}\PY{p}{(}\PY{n}{image}\PY{o}{.}\PY{n}{reshape}\PY{p}{(}\PY{n}{img\PYZus{}size}\PY{p}{,} \PY{n}{img\PYZus{}size}\PY{p}{,} \PY{n}{num\PYZus{}channels}\PY{p}{)}\PY{p}{,}
                        \PY{n}{interpolation}\PY{o}{=}\PY{l+s+s1}{\PYZsq{}}\PY{l+s+s1}{nearest}\PY{l+s+s1}{\PYZsq{}}\PY{p}{)}
             \PY{n}{plt}\PY{o}{.}\PY{n}{show}\PY{p}{(}\PY{p}{)}
\end{Verbatim}


    Plot an image from the test-set which will be used as an example below.

    \begin{Verbatim}[commandchars=\\\{\}]
{\color{incolor}In [{\color{incolor}62}]:} \PY{n}{image1} \PY{o}{=} \PY{n}{test\PYZus{}images}\PY{p}{[}\PY{l+m+mi}{0}\PY{p}{]}
         \PY{n}{plot\PYZus{}image}\PY{p}{(}\PY{n}{image1}\PY{p}{)}
\end{Verbatim}


    \begin{center}
    \adjustimage{max size={0.9\linewidth}{0.9\paperheight}}{output_114_0.png}
    \end{center}
    { \hspace*{\fill} \\}
    
    Plot another example image from the test-set.

    \begin{Verbatim}[commandchars=\\\{\}]
{\color{incolor}In [{\color{incolor}63}]:} \PY{n}{image2} \PY{o}{=} \PY{n}{test\PYZus{}images}\PY{p}{[}\PY{l+m+mi}{13}\PY{p}{]}
         \PY{n}{plot\PYZus{}image}\PY{p}{(}\PY{n}{image2}\PY{p}{)}
\end{Verbatim}


    \begin{center}
    \adjustimage{max size={0.9\linewidth}{0.9\paperheight}}{output_116_0.png}
    \end{center}
    { \hspace*{\fill} \\}
    
    \subsubsection{Convolution Layer 1}\label{convolution-layer-1}

    Now plot the filter-weights for the first convolutional layer.

Note that positive weights are red and negative weights are blue.

    \begin{Verbatim}[commandchars=\\\{\}]
{\color{incolor}In [{\color{incolor}64}]:} \PY{n}{plot\PYZus{}conv\PYZus{}weights}\PY{p}{(}\PY{n}{weights}\PY{o}{=}\PY{n}{weights\PYZus{}conv1}\PY{p}{)}
\end{Verbatim}


    \begin{center}
    \adjustimage{max size={0.9\linewidth}{0.9\paperheight}}{output_119_0.png}
    \end{center}
    { \hspace*{\fill} \\}
    
    Applying each of these convolutional filters to the first input image
gives the following output images, which are then used as input to the
second convolutional layer. Note that these images are down-sampled to
about half the resolution of the original input image.

    \begin{Verbatim}[commandchars=\\\{\}]
{\color{incolor}In [{\color{incolor}65}]:} \PY{n}{plot\PYZus{}conv\PYZus{}layer}\PY{p}{(}\PY{n}{layer}\PY{o}{=}\PY{n}{layer\PYZus{}conv1}\PY{p}{,} \PY{n}{image}\PY{o}{=}\PY{n}{image1}\PY{p}{)}
\end{Verbatim}


    \begin{center}
    \adjustimage{max size={0.9\linewidth}{0.9\paperheight}}{output_121_0.png}
    \end{center}
    { \hspace*{\fill} \\}
    
    The following images are the results of applying the convolutional
filters to the second image.

    \begin{Verbatim}[commandchars=\\\{\}]
{\color{incolor}In [{\color{incolor}66}]:} \PY{n}{plot\PYZus{}conv\PYZus{}layer}\PY{p}{(}\PY{n}{layer}\PY{o}{=}\PY{n}{layer\PYZus{}conv1}\PY{p}{,} \PY{n}{image}\PY{o}{=}\PY{n}{image2}\PY{p}{)}
\end{Verbatim}


    \begin{center}
    \adjustimage{max size={0.9\linewidth}{0.9\paperheight}}{output_123_0.png}
    \end{center}
    { \hspace*{\fill} \\}
    
    \subsubsection{Convolution Layer 2}\label{convolution-layer-2}

    Now plot the filter-weights for the second convolutional layer.

There are 16 output channels from the first conv-layer, which means
there are 16 input channels to the second conv-layer. The second
conv-layer has a set of filter-weights for each of its input channels.
We start by plotting the filter-weigths for the first channel.

Note again that positive weights are red and negative weights are blue.

    \begin{Verbatim}[commandchars=\\\{\}]
{\color{incolor}In [{\color{incolor}67}]:} \PY{n}{plot\PYZus{}conv\PYZus{}weights}\PY{p}{(}\PY{n}{weights}\PY{o}{=}\PY{n}{weights\PYZus{}conv2}\PY{p}{,} \PY{n}{input\PYZus{}channel}\PY{o}{=}\PY{l+m+mi}{0}\PY{p}{)}
\end{Verbatim}


    \begin{center}
    \adjustimage{max size={0.9\linewidth}{0.9\paperheight}}{output_126_0.png}
    \end{center}
    { \hspace*{\fill} \\}
    
    There are 16 input channels to the second convolutional layer, so we can
make another 15 plots of filter-weights like this. We just make one more
with the filter-weights for the second channel.

    \begin{Verbatim}[commandchars=\\\{\}]
{\color{incolor}In [{\color{incolor}68}]:} \PY{n}{plot\PYZus{}conv\PYZus{}weights}\PY{p}{(}\PY{n}{weights}\PY{o}{=}\PY{n}{weights\PYZus{}conv2}\PY{p}{,} \PY{n}{input\PYZus{}channel}\PY{o}{=}\PY{l+m+mi}{1}\PY{p}{)}
\end{Verbatim}


    \begin{center}
    \adjustimage{max size={0.9\linewidth}{0.9\paperheight}}{output_128_0.png}
    \end{center}
    { \hspace*{\fill} \\}
    
    It can be difficult to understand and keep track of how these filters
are applied because of the high dimensionality.

Applying these convolutional filters to the images that were ouput from
the first conv-layer gives the following images.

Note that these are down-sampled yet again to half the resolution of the
images from the first conv-layer.

    \begin{Verbatim}[commandchars=\\\{\}]
{\color{incolor}In [{\color{incolor}69}]:} \PY{n}{plot\PYZus{}conv\PYZus{}layer}\PY{p}{(}\PY{n}{layer}\PY{o}{=}\PY{n}{layer\PYZus{}conv2}\PY{p}{,} \PY{n}{image}\PY{o}{=}\PY{n}{image1}\PY{p}{)}
\end{Verbatim}


    \begin{center}
    \adjustimage{max size={0.9\linewidth}{0.9\paperheight}}{output_130_0.png}
    \end{center}
    { \hspace*{\fill} \\}
    
    And these are the results of applying the filter-weights to the second
image.

    \begin{Verbatim}[commandchars=\\\{\}]
{\color{incolor}In [{\color{incolor}70}]:} \PY{n}{plot\PYZus{}conv\PYZus{}layer}\PY{p}{(}\PY{n}{layer}\PY{o}{=}\PY{n}{layer\PYZus{}conv2}\PY{p}{,} \PY{n}{image}\PY{o}{=}\PY{n}{image2}\PY{p}{)}
\end{Verbatim}


    \begin{center}
    \adjustimage{max size={0.9\linewidth}{0.9\paperheight}}{output_132_0.png}
    \end{center}
    { \hspace*{\fill} \\}
    
    \subsubsection{Write Test Predictions to
CSV}\label{write-test-predictions-to-csv}

    \begin{Verbatim}[commandchars=\\\{\}]
{\color{incolor}In [{\color{incolor}71}]:} \PY{c+c1}{\PYZsh{} def write\PYZus{}predictions(ims, ids):}
         \PY{c+c1}{\PYZsh{}     ims = ims.reshape(ims.shape[0], img\PYZus{}size\PYZus{}flat)}
         \PY{c+c1}{\PYZsh{}     preds = session.run(y\PYZus{}pred, feed\PYZus{}dict=\PYZob{}x: ims\PYZcb{})}
         \PY{c+c1}{\PYZsh{}     result = pd.DataFrame(preds, columns=classes)}
         \PY{c+c1}{\PYZsh{}     result.loc[:, \PYZsq{}id\PYZsq{}] = pd.Series(ids, index=result.index)}
         \PY{c+c1}{\PYZsh{}     pred\PYZus{}file = \PYZsq{}predictions.csv\PYZsq{}}
         \PY{c+c1}{\PYZsh{}     result.to\PYZus{}csv(pred\PYZus{}file, index=False)}
         
         \PY{c+c1}{\PYZsh{} write\PYZus{}predictions(test\PYZus{}images, test\PYZus{}ids)}
\end{Verbatim}


    \subsubsection{Close TensorFlow Session}\label{close-tensorflow-session}

    We are now done using TensorFlow, so we close the session to release its
resources.

    \begin{Verbatim}[commandchars=\\\{\}]
{\color{incolor}In [{\color{incolor}72}]:} \PY{n}{session}\PY{o}{.}\PY{n}{close}\PY{p}{(}\PY{p}{)}
\end{Verbatim}



    % Add a bibliography block to the postdoc
    
    
    
    \end{document}
